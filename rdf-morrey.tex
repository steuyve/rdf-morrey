\documentclass[12pt]{amsart}

\textwidth=5.5in \textheight=8.5in
\usepackage{cancel}
\usepackage{latexsym, amssymb, amsmath}
\usepackage{soul,esint}
\usepackage{amsfonts, graphicx}
\usepackage{graphicx,color}
\usepackage[backend=biber,style=science,sorting=nyt]{biblatex}
%\nocite{*}
\usepackage{amsthm}
\usepackage{hyperref}
\usepackage{enumerate}
\usepackage{color}
\usepackage{verbatim}
%\usepackage{refcheck}

\theoremstyle{plain}
\newtheorem{theorem}[subsection]{Theorem}
\theoremstyle{plain}
\newtheorem{proposition}[subsection]{Proposition}
\newtheorem{lemma}[subsection]{Lemma}
\newtheorem{corollary}[subsection]{Corollary}

\theoremstyle{definition}
\newtheorem{definition}[subsection]{Definition}

\theoremstyle{remark}
\newtheorem{remark}[subsection]{Remark}

\numberwithin{equation}{subsection}

\addbibresource{rdf-morrey.bib}

\newcommand{\del}{\nabla}
\newcommand{\hdel}{\tilde{\nabla}}

\begin{document}
\title[]
{Ricci-DeTurck Flow from Initial Metric with Morrey-type Integrability Condition}

\author{Man-Chun Lee}
\address[Man-Chun Lee]{Room 237, Lady Shaw Building,
The Chinese University of Hong Kong,
Shatin, N.T., Hong Kong}
\email{mclee@math.cuhk.edu.hk}

\author{Stephen Shang Yi Liu}
\address[Stephen Shang Yi Liu]{Room 222A, Lady Shaw Building,
The Chinese University of Hong Kong,
Shatin, N.T., Hong Kong}
 \email{syliu@math.cuhk.edu.hk}

\renewcommand{\subjclassname}{
  \textup{2010} Mathematics Subject Classification}
\subjclass[2010]{Primary 53E20
}

\date{\today}

\begin{abstract}
    Introduced by Hamilton, the Ricci flow has seen a number of successful applications to problems in geometry. In recent years, various short-time existence theory for Ricci flow from initial metrics with lower regularity has been studied, with applications to scalar curvature problems. Motivated by the recent work of Chu and Lee in \cite{chu_ricci-deturck_2022}, we study the short-time existence theory of Ricci-DeTurck flow starting from rough metrics which are bi-Lipschitz and satisfy a Morrey-type integrability condition. Using the rough existence theory, we are able to show the preservation and improvement of distributional scalar curvature lower bounds in the sense of \cite{lee_positive_2015} provided the singular set for such metrics is not too large. Our result supplements those of Jiang-Sheng-Zhang in \cite{jiang_weak_2021} and \cite{jiang_removable_2022}.
\end{abstract}

\keywords{Ricci-DeTurck flow, scalar curvature}

\maketitle

%\newpage
%
%
%\renewcommand*\contentsname{Table of Contents}
%{
%\setcounter{tocdepth}{3}
%\tableofcontents
%}
%
%\newpage

\section{Introduction}\label{sec:introduction}

Let $M^n$ be a Riemannian manifold. Then the Ricci flow is a family of metrics $h(t)$ evolving in the direction of their Ricci tensors,

\begin{equation*}
    \frac{\partial}{\partial t} h(t) = -2\text{Ric}(h(t)).
\end{equation*}

Introduced by Hamilton in \cite{hamilton_three-manifolds_1982}, the Ricci flow has seen a number of successful applications to a number of problems in geometry, most famously in Perelman's resolution of Thurston's Geometrization conjecture. It is a weakly parabolic system, and in \cite{deturck_deforming_1983}, DeTurck showed that it is diffeomorphic to a strictly parabolic system, the Ricci-DeTurck $h$-flow, which is easier to study from a PDE point of view.

Let $\text{Sym}_2(T^\ast M)$ denote the space of symmetric 2-tensors on $M$ and let $X:\text{Sym}_2(T^\ast M) \to TM$ be given by
\begin{equation*}
    X_{h}(g)^k := g^{ij}\left(\tilde{\Gamma}^k_{ij} - \Gamma^k_{ij}\right)
\end{equation*}
where $\tilde{\Gamma}$ and $\Gamma$ are the Christoffel symbols of $h$ and $g$ respectively. Then the Ricci-DeTurck $h$-flow (here using the terminology as in \cite{simon_deformation_2002} to emphasize the dependance on background metric $h$) is given by the equation
\begin{equation}\label{eqn:prelim-ricci-deturck-flow}
    \frac{\partial}{\partial t} g(t) = -2\text{Ric}(g(t)) - \mathcal{L}_{X_{h(t)}(g(t))}g(t)
\end{equation}
Solutions to the Ricci-DeTurck $h$-flow are related to a Ricci flow via pullback by diffeomorphisms, that is, if $g(t)$ solves (\ref{eqn:prelim-ricci-deturck-flow}) and $\chi_t:M\to M$ is a family of diffeomorphisms satisfying
\begin{equation}\label{eqn:prelim-rdf-to-rf-diffeomorphisms}
    \begin{cases}
        &X_{h(t)}(g(t))f = \frac{\partial}{\partial t}\left(f \circ \chi_t\right) \text{ for all } f \in C^\infty(M) \\
        &\chi_0 = \text{id},
    \end{cases}
\end{equation}
then $\chi_t^\ast g(t)$ is a Ricci flow with initial data $g(0)$.

In local coordinates, (\ref{eqn:prelim-ricci-deturck-flow}) is given by
\begin{align}\label{eqn:prelim-rdf-local-coords}
    \frac{\partial}{\partial t} g_{ij} &= g^{pq}\hdel_p\hdel_qg_{ij} - g^{k\ell}g_{ip}h^{pq}\tilde{R}_{jkq\ell} - g^{k\ell}g_{jp}h^{pq}\tilde{R}_{ikq\ell} \nonumber \\
    &+ \frac{1}{2}g^{k\ell}g^{pq}\left(\hdel_i g_{pk}\hdel_j g_{q\ell} + 2\hdel_k g_{jp}\hdel_q g_{i\ell} - 2\hdel_k g_{jp} \hdel_\ell g_{iq} \right. \nonumber \\
    &\left. -2\hdel_j g_{pk} \hdel_\ell g_{iq} - 2\hdel_i g_{pk} \hdel_\ell g_{qj}\right),
\end{align}
where $\hdel$ and $\tilde{R}$ denote the Levi-Civita connection and Riemann curvature tensor respectively of $h$ and we are suppressing the dependance on $t$ for $g(t)$ in the notation above.

Since the flow is a parabolic system, one usually expects that the flow will improve the regularity of initial data. In view of this, there has been a wide variety of literature extending the existence theory of Ricci Flow and Ricci-DeTurck $h$-flow to settings with less regularity. The foundational result in this area is by Shi in \cite{shi_deforming_1989} who showed the short-time existence of a solution to the Ricci-DeTurck flow from a metric that is complete with bounded curvature. A metric $g$ is called $L^\infty$ or bi-Lipschitz if $\Lambda^{-1}h \leq g \leq \Lambda h$ almost everywhere on $M$ with $\Lambda >1$ and $h$ is some fixed smooth background metric. Building on the result of Shi, Simon in \cite{simon_deformation_2002} established the following regularization result: if $(1+\delta)^{-1}h \leq g_0 \leq (1+\delta)h$ for some sufficiently small dimensional constant $\delta$, then there exists a short-time solution to the Ricci-DeTurck $h$-flow on $M \times (0,T]$ which is smooth for $t > 0$ with convergence back to $g_0$ in $C^0_\text{loc}$ if $g_0$ was additionally in $C^0_\text{loc}$. In Euclidean space, Koch and Lamm in \cite{koch_geometric_2012} used heat kernel estimates to show the existence and uniqueness of a global and analytic solution to the Ricci-DeTurck $h$-flow on $\mathbb{R}^n$ from $L^\infty$ initial metrics that are close in $L^\infty$ to the Euclidean metric. Building on these results, Burkhardt-Guim in \cite{burkhardt-guim_pointwise_2019} further studied Ricci flow from $C^0$ initial metrics on closed manifolds, in particular relying on heat kernel estimates to establish an iteration scheme to construct a short-time solution and applied this Ricci flow existence theory to study scalar curvature lower bounds and a question of Gromov. In the direction of more general $L^\infty$ initial data, Lamm and Simon in \cite{lamm_ricci_2021} showed the short-time existence for the Ricci-DeTurck $h$-flow on complete four-manifolds for rough initial metrics that are in $L^\infty \cap W^{2,2}$. They showed that solutions are uniformly smooth for positive time and converge back to the initial data in $W^{2,2}_\text{loc}$ sense as $t\to0$. In \cite{chu_ricci-deturck_2022}, Chu and Lee developed this further and considered the case where $g_0 \in L^\infty \cap W^{1,n}$, showing the existence of the flow starting from metrics which are $L^\infty$ and satisfy a small local gradient concentration in a $W^{1,n}$ sense, and study a number of applications.

Motivated by the work of Chu and Lee, and the heat kernel based approach of Burkhardt-Guim, in this work we consider the existence theory for metrics that are in $L^\infty$ and satisfy a $L^1$ Morrey-type integrability condition (condition (ii) of Theorem \ref{thm:intro-main-theorem} below) and study a number of applications. We will see that the Morrey-type condition allows for the use of heat kernel estimates to establish a priori estimates that are important for the existence theory. Our main theorem is the following existence theorem for the smooth $h$-flow with quantitative estimates.

\begin{theorem}\label{thm:intro-main-theorem}
    Let $g_0$ be a smooth metric on $M^n$ and $h$ a fixed smooth background metric on $M$ and suppose for each $m \in \mathbb{N}$, there is a $C_m = C(n,m) > 0$ such that $|\del^m\text{Rm}(h)| \leq C_m$. Suppose $g_0$ satisfies the following:
    \begin{enumerate}[(i)]
        \item there is a $\Lambda_0 > 0$ such that $\Lambda_0^{-1}h \leq g_0 \leq \Lambda_0 h$ on $M$;
        \item for $p \geq 1, \delta > 0$, there is a $L_0 > 0, r_0 > 0$ such that for all $x_0 \in M$ and $0 < r < r_0$,
        \begin{equation*}
            \fint_{B(x_0,r)}|\hdel g_0|^p d\mu_h \leq L_0 r^{-p+\delta}.
        \end{equation*}
    \end{enumerate}
    Then there are $T(n, \Lambda_0, L_0, p, \delta, r_0), L(n, \Lambda_0, L_0, p, \delta, r_0) > 0$, $\Lambda(n, \Lambda_0, L_0, p, \delta, r_0) > 1$ and a smooth solution $g(t)$ to the Ricci-DeTurck $h$-flow on $M\times(0,T]$ such that:
        \begin{enumerate}[(a)]
            \item $\Lambda^{-1}h\leq g(t) \leq \Lambda h$ on $M\times(0,T]$;
            \item for all $x_0 \in M$, $0 < r < r_0$, $t \in (0, T]$,
            \begin{equation*}
                \fint_{B(x_0,r)} |\hdel g(t)|^p d\mu_h \leq L r^{-p+\delta};
            \end{equation*}
            \item for any $k \in \mathbb{N}$, there is $B_k(n, k, \Lambda_0, L_0, p, \delta, R) > 0$ such that for all $t \in (0, T]$, $x_0 \in M$,
            \begin{equation*}
                \sup\limits_{B(x_0, R)}|\hdel^k g(t)| \leq \frac{B_k}{t^{\frac{1}{2}(m-\frac{\delta}{p})}},
            \end{equation*}
            with $B_k$ decreasing (i.e. improving) as $R$ increases.
        \end{enumerate}
\end{theorem}

As a first application, we obtain the following theorem to rough initial metrics on complete non-compact manifolds. We also obtain a number of other applications that will be explored further in the last section. The rough existence theory is given in the following theorem.

\begin{theorem}\label{thm:intro-application-1-statement}
    Suppose $(M^n, h)$ is a complete Riemannian manifold and suppose for each $m \in \mathbb{N}$, there is a $C_m = C(n,m) > 0$ such that $|\del^m\text{Rm}(h)| \leq C_m$. Suppose $g_0$ is a Riemannian metric on $M$ and $\Sigma \subseteq M$ is a compact set so that the following holds:
    \begin{enumerate}[(i)]
        \item $g_0$ is not necessarily smooth on $\Sigma$ but $g_0$ is smooth on $M \setminus \Sigma$;
        \item there is a $\Lambda_0 > 1$ such that $\Lambda_0^{-1} h \leq g_0 \leq \Lambda_0 h$ on $M$;
        \item for $p \geq 1$, $\delta > 0$, there is a $L_0 > 0$, $r_0 > 0$ such that for every $x_0 \in M$, $0 < r < r_0$,
        \begin{equation*}
            \fint_{B(x_0,r)} |\hdel g_0|^p d\mu_h \leq L_0 r^{-p+\delta}.
        \end{equation*}
    \end{enumerate}
    Then there are $T(n, \Lambda_0, L_0, p, \delta, r_0), L(n, \Lambda_0, L_0, p, \delta, r_0) > 0$, $ \Lambda(n, \Lambda_0, L_0, p, \delta, r_0) > 1$ and a smooth solution $g(t)$ to the Ricci-DeTurck $h$-flow on $M\times(0,T]$ such that:
        \begin{enumerate}[(a)]
            \item $\Lambda^{-1}h\leq g(t) \leq \Lambda h$ on $M\times(0,T]$;
            \item for all $x_0 \in M$, $0 < r < r_0$, $t \in (0, T]$,
            \begin{equation*}
                \fint_{B(x_0,r)} |\hdel g(t)|^p d\mu_h \leq L r^{-p+\delta};
            \end{equation*}
            \item for any $k \in \mathbb{N}$, there is $B_k(n, k, \Lambda_0, L_0, p, \delta, R) > 0$ such that for all $t \in (0, T]$, $x_0 \in M$,
            \begin{equation*}
                \sup\limits_{B(x_0, R)}|\hdel^k g(t)| \leq \frac{B_k}{t^{\frac{1}{2}(m-\frac{\delta}{p})}},
            \end{equation*}
            with $B_k$ decreasing (i.e. improving) as $R$ increases;
            \item $g(t) \to g_0$ in $C_\text{loc}^0(M)$ as $t \to 0$;
            \item $g(t) \to g_0$ in $C_\text{loc}^\infty(M\setminus\Sigma)$ as $t \to 0$.
        \end{enumerate}
\end{theorem}

We will see in Lemma \ref{lem:estimates-h-star} that we can produce a metric $\tilde{h}$ with uniformly bounded curvature of all orders as assumed in Theorems \ref{thm:intro-main-theorem} and \ref{thm:intro-application-1-statement} by a small perturbation of a metric $h$ only satisfying $|\text{Rm}(h)|\leq 1$. So, as we will note in  Remark \ref{rmk:estimates-h-remark}, it will be enough to assume that $|\text{Rm}(h)|\leq 1$ and then work with the metric $\tilde{h}$.

In \cite{lee_positive_2015}, Lee--Lefloch introduced a notion of distributional scalar curvature for metrics with lower regularity. In \cite{jiang_removable_2022}, the authors showed that for metrics $g \in C^0 \cap W^{1,p}_{\text{loc}}(M)$ where $M$ is complete asymptotically flat with $n \leq p \leq \infty$ satisfying $R_g \geq 0$ outside some compact singular set $\Sigma$ can be shown to have non-negative distributional scalar curvature on all of $M$ provided the singular set is not too large in terms of Hausdorff measure. In \cite{jiang_weak_2021}, the authors also show that distributional scalar curvature lower bounds can be preserved and improved to classical scalar curvature lower bounds along Ricci flow for metrics in $W^{1,p}(M)$ for $n < p \leq \infty$ where $M$ is compact. These results are related to positive mass theorems for lower regularity metrics and a conjecture of Schoen regarding the removable singularity of $L^\infty$ metrics on $\mathbb{T}^n$ with co-dimension three singularity and $R_g \geq 0$ on the regular part. In Theorem \ref{thm:preservation-distributional-scalar-curvature-lower-bound} and Corollary \ref{thm:non-cpt-dist-scalar-curvature-lower-bdd}, we use the rough existence theory established in Theorem \ref{thm:intro-application-1-statement} to obtain similar results for metrics which are $L^\infty$ and satisfy the Morrey-type condition, as long as the singular set is small in terms of a notion of co-dimension introduced by Lee and Tam in \cite{lee_continuous_2021}.

We now outline the rest of this work. In Section \ref{sec:a-priori-estimates}, we establish the priori estimates that will be used later. In Section \ref{sec:proof-of-main-theorem}, we prove Theorem \ref{thm:intro-main-theorem}. Finally, in Section \ref{sec:applications}, we study the applications outlined above.

\section{A Priori Estimates}\label{sec:a-priori-estimates}

In this chapter we establish quantitative estimates that will be important in the proof of the main theorem.

\subsection{Preparatory Lemmata}

Before we move on to the estimates in earnest, we first establish a few preparatory lemmata.

We first mention the following proposition of Chu-Lee (Proposition 2.1 of \cite{chu_ricci-deturck_2022}) regarding the smooth background metric $h$, which shows that starting from the assumption that $h$ has curvature bound, we can assume without loss of generality that all higher derivatives of $\text{Rm}(h)$ are uniformly bounded.

\begin{lemma}[Proposition 2.1 of \cite{chu_ricci-deturck_2022}]\label{lem:estimates-h-star}
    Suppose $(M, h_0)$ is a complete bounded manifold with $|\text{Rm}(h_0)|\leq 1$, then for any $\varepsilon > 0$, there is a complete metric $\tilde{h}$ on $M$ such that
    \begin{enumerate}
        \item $\lVert \tilde{h} - h_0\rVert_{C^1(M,h_0)} < \varepsilon$;
        \item for all $k \in \mathbb{N}$, there is a $C_1(n,k,\varepsilon) > 0$ such that
        \begin{equation*}
            \sup\limits_M |\del^k \text{Rm}(\tilde{h})| \leq C_1.
        \end{equation*}
    \end{enumerate}
\end{lemma}

Here the parentheses $C_1(n,k,\varepsilon) > 0$ means that $C_1$ is a positive constant depending on the quantities $n, k, \varepsilon$. We will use similar notation throughout this work. The proof can be found in \cite{chu_ricci-deturck_2022}.

\begin{remark}\label{rmk:estimates-h-remark}
With this lemma in place, given any smooth background metric $h_0$ with curvature bound, by scaling so that $|\text{Rm}(h_0)|\leq 1$, we may as well assume $h$ (taking $h$ to be the $\tilde{h}$ above) satisfies the following
\begin{equation}\label{eqn:h-remark-curvature-estimates}
    \forall k \in \mathbb{N}, \exists C_k(n,k) > 0, \sup\limits_{M}|\del^k\text{Rm}(h)| \leq C_k
\end{equation}
\end{remark}

In the following and the rest of this work, we denote by $k_i := \sup\limits_M|\hdel^i\text{Rm}(h)|$ and, as mentioned above, we will specify using parentheses the quantities that certain numbered constants depend upon, e.g. $C_j = C_j(n,k_1,\dots,k_i) > 0$ denotes a positive constant $C_j$ that depends on the dimension and quantities $k_1$, up to $k_i$. These constants may be changing line by line throughout and we will only re-introduce the parentheses when the dependency changes.

Following \cite{shi_deforming_1989} and \cite{simon_deformation_2002}, the next lemma shows that given a first order interior estimate, we may parabolically bootstrap to obtain higher order interior estimates.

\begin{lemma}\label{lem:estimates-bootstrap}
    Suppose $g(t)$ is a smooth solution to the Ricci-DeTurck $h$-flow on $M \times [0,T]$ for some smooth background metric $h$ satisfying (\ref{eqn:h-remark-curvature-estimates}) as in Remark \ref{rmk:estimates-h-remark} so that
    \begin{enumerate}[(i)]
        \item $\Lambda^{-1}h \leq g(t) \leq \Lambda h$ on $M\times[0,T]$ for some $\Lambda > 1$;
        \item let $R > 0, \gamma > 0$, then there are $B_1 = B_1(n,R,\gamma,\Lambda,k_1), T_1 = T_1(n,\Lambda,p,\delta) > 0$ such that
        \begin{equation*}
            |\hdel g(x,t)| \leq \frac{B_1}{t^{\frac{1}{2}-\frac{\delta}{2p}}}
        \end{equation*}
        for all $x \in B\left(x_0, R + \frac{\gamma}{2}\right), t \in (0, T_1]$. Here and throughout, the balls are measured with respect to the metric $h$, unless noted otherwise.
    \end{enumerate}
    Let $m \in \mathbb{N}$. Then there are constants 
    \begin{equation*}
        B_m = B_m(n,R,\gamma, m,\Lambda,k_1,\dots,k_m) > 0
    \end{equation*}
    such that 
    \begin{equation}\label{eqn:bootstrap-higher-order-interior-estimate}
        \sup\limits_{B\left(x_0, R + \frac{\gamma}{2^{m+1}}\right)} |\hdel^m g(x,t)| \leq \frac{B_m}{t^{\frac{1}{2}(m-\frac{\delta}{p})}}
    \end{equation}
    for all $t \in (0, T_1]$. Note that as $m \to \infty$, the radii converge to $R$.
\end{lemma}

\begin{proof}[Sketch of Proof]
    The proof of the above lemma is a standard induction argument, using the non-scaling invariant order on $t$ given in assumption (ii) above instead of the $t^{-\frac{1}{2}}$ found in Lemma 4.1-4.2 of \cite{simon_deformation_2002}. For $m \in \mathbb{N}$, we obtain (\ref{eqn:bootstrap-higher-order-interior-estimate}) by calculating the evolution equation under the operator $\partial_t - g^{ij}\hdel_i\hdel_j$ of 
    \begin{equation*}
        t^{2m-1-\frac{2\delta}{p}}|\hdel^m g|^2\left(|\hdel^{m-1}g|^2 + Lt^{-m+1+\frac{\delta}{p}}\right),
    \end{equation*}
    multiplying by a cut-off function first as necessary.
\end{proof}

The next proposition shows that smooth solutions to the Ricci-DeTurck $h$-flow locally preserve smoothness up to $t=0$ as long as the initial metric $g_0$ was locally smooth.

\begin{proposition}\label{prop:time-zero-smoothness}
    Suppose $g(t)$ is a smooth solution to the Ricci-DeTurck $h$-flow on $M \times [0,T]$ for some smooth background metric $h$ satisfying (\ref{eqn:h-remark-curvature-estimates}) as in Remark \ref{rmk:estimates-h-remark} and let $\Omega \Subset M$ so that
    \begin{enumerate}[(i)]
        \item there is an $\varepsilon = \varepsilon(n) > 0$ such that $(1-\varepsilon)h \leq g(t) \leq (1+\varepsilon)h$ on $\Omega\times[0,T]$;
        \item there are constants $C_i > 0$ such that $\sup\limits_{\Omega}\sum\limits_{m=1}^i |\hdel^m g_0|\leq C_i$.
    \end{enumerate}
    Then for all $\Omega' \Subset \Omega$,
    \begin{equation*}
        \sup\limits_{\Omega' \times [0,T]} |\hdel^i g(t)| \leq C
    \end{equation*}
    for some $C > 0$ depending on $n, \Lambda, i, k_0, \dots, k_i, C_1, \dots, C_i, \Omega', \Omega$.
\end{proposition}

The proposition is essentially Lemmas 4.1 and 4.2 of \cite{shi_deforming_1989}. See also \cite{simon_deformation_2002}, Proposition 2.2 of \cite{lee_continuous_2021} and Proposition 2.2 of \cite{chu_ricci-deturck_2022} for similar results in other settings. Note that condition (i) above differs from our $L^\infty$ condition in our setting. We will see in the proof of Theorem \ref{thm:intro-application-1-statement} how we adapt this proposition to our setting.

\subsection{Gradient Estimate}

We now show that the interior gradient estimate assumed in Lemma \ref{lem:estimates-bootstrap} can be obtained for smooth solutions of the Ricci-DeTurck $h$-flow under the $L^\infty$ and Morrey-type integrability assumption.

\begin{lemma}\label{lem:gradient-estimate}
    Suppose $g(t)$ is a smooth solution to the Ricci-DeTurck $h$-flow on $M \times [0,T]$ for some smooth initial background metric $h$ satisfying (\ref{eqn:h-remark-curvature-estimates}) as in Remark \ref{rmk:estimates-h-remark} so that
    \begin{enumerate}[(i)]
        \item $\Lambda^{-1}h \leq g(t) \leq \Lambda h$ on $M\times[0,T]$ for some $\Lambda > 1$;
        \item for some $p \geq 1, \delta > 0$, there are $L_0, r_0 > 0$ such that for all $x_0 \in M$ and $0 < r < r_0$,
        \begin{equation*}
            \fint_{B(x_0,r)} |\hdel g(x,0)|^p d\mu_h(x) \leq L_0r^{-p+\delta}.
        \end{equation*}
    \end{enumerate}
    Then there are $B_1 = B_1(n,\Lambda,L_0,k_1), T_1 = T_1(n,\Lambda,L_0,p,\delta) > 0$ such that
    \begin{equation}\label{eqn:interior-gradient-estimate}
        \sup\limits_M|\hdel g(x,t)| \leq \frac{B_1}{t^{\frac{1}{2}-\frac{\delta}{2p}}}
    \end{equation}
    for all $x \in M, t \in (0, T_1]$.
\end{lemma}

\begin{proof}
    For the sake of convenience, we will suppress the dependence on $t$ and denote $g(t)$ by $g$ unless explicitly stated otherwise.
    
    Let $T_1$ be the maximal time on which (\ref{eqn:interior-gradient-estimate}) holds, that is, there is a $x_0 \in M$ such that
    \begin{equation}\label{eqn:gradient-estimate-maximal-T1}
        \begin{cases}
            &\sup\limits_M |\hdel g(x,t)| \leq \frac{B_1}{t^{\frac{1}{2}-\frac{\delta}{2p}}} \text{ on } (0,T_1) \text{ and, } \\
            &|\hdel g(x_0, T_1)| = \frac{B_1}{T_1^{\frac{1}{2}-\frac{\delta}{2p}}}.
        \end{cases}
    \end{equation}
    By rescaling, we can assume without loss of generality that $T_1 \leq 1$. It suffices to estimate $T_1$ from below. 
    
    Our first step is to compute the evolution equation of $|\hdel g|^2$ under the operator $\Box_g := \frac{\partial}{\partial t} - \Delta^{g} - \nabla_X$ where $\Delta^g = g^{ij}\del_i\del_j$ is taking derivatives with respect to the metric $g$ instead of $h$. By (\ref{eqn:prelim-ricci-deturck-flow}) and Young's inequality, there are $C_1, C_2, C_3 > 0$ depending only on $n, \Lambda, k_1$ such that
    \begin{align}\label{eqn:gradient-estimate-partialt-norm-hdelg}
        \frac{\partial}{\partial t} |\hdel g|^2 \leq g^{ij}\hdel_i\hdel_j|\hdel g|^2 - C_1|\hdel^2 g|^2 + C_2 |\hdel g|^4 + C_3.
    \end{align}
    Since $|\hdel g|$ may not be smooth, define the smooth function $Q$ by
    \begin{equation*}
        Q := \left(|\hdel g|^2 + \sigma^2\right)^{\frac{1}{2}}
    \end{equation*}
    for $\sigma \in (0, 1)$. By Kato's inequality, we have that $|\hdel Q^2|\leq |\hdel^2 g|$. Using the fact that for any function $f$, locally
    \begin{equation*}
        g^{ij}\hdel_i\hdel_j f = \Delta^g f - X^kf_k
    \end{equation*}
    and that under assumption (i) above,
    \begin{equation*}
        |X|\leq C|\hdel g|,\qquad \del_X f \leq C|X||\del f|
    \end{equation*}
    for a constant $C = C(n,\Lambda) > 0$, (\ref{eqn:gradient-estimate-partialt-norm-hdelg}) becomes
    \begin{align}\label{eqn:gradient-estimate-partialt-Qsquared}
        \frac{\partial}{\partial t} Q^2 &\leq \Delta^g Q^2 + \del_X Q^2 + C_2Q^4 + C_3.
    \end{align}
    In view of the maximum principle which we hope to apply, we can assume that $Q$ admits a uniform lower bound on $B(x_0, R)$. If that is not the case then $Q$ is uniformly bounded from above and we are already done. So we have
    \begin{equation}\label{eqn:gradient-estimate-box-Q2}
        \Box_g Q \leq C_2 Q^3 + C_3.
    \end{equation}
    Let $f(t) = \exp\left(-A(t^{\frac{\delta}{p}} + \sigma^2 t)\right)$ for a constant $A > 0$ to be determined. For all $t < T_1$, (\ref{eqn:gradient-estimate-box-Q2}) and (\ref{eqn:gradient-estimate-maximal-T1}) imply in the sense of barrier that in $B(x_0, R)$,
    \begin{align}\label{eqn:gradient-estimate-box-Q3}
        \Box_g\left(f(t)Q-C_3t\right) &= f(t)\left(-\frac{A\delta}{p}t^{\frac{\delta}{p}-1} - A\sigma^2\right)Q + f(t)\Box_g Q - C_3 \nonumber \\
        &\leq f(t)C_3 - C_3 \nonumber \\
        &\leq 0
    \end{align}
    by choosing $A$ sufficiently large in the second inequality above and using the fact that $f(t) \leq 1$ for all $t \geq 0$ for the last inequality.
    Hence, by the maximum principle we obtain
    \begin{equation}\label{eqn:gradient-estimate-invoke-heat-kernel}
        f(T_1)B_1T_1^{-\frac{1}{2}+\frac{\delta}{2p}} - C_3T_1 \leq \int_{M} K(x_0, T_1; y, 0)f(0)Q(y,0)d\mu_h(y)
    \end{equation}
    where $K$ is the heat kernel for the heat operator $\Box_g$.
    
    We now wish to obtain an estimate for the heat kernel $K$. We do this by modifying an estimate established by Bamler--Cabezas-Rivas--Wilking in \cite{bamler_ricci_2017}. Let $\widetilde{G}(x,t;y,s)$ denote the heat kernel associated to the backwards heat equation coupled with a Ricci flow $h(t)$. That is,
    \begin{equation*}
        \left(\partial_s + \Delta^{h_t}_{y,s}\right)\widetilde{G}(x,t;\cdot,\cdot) = 0,\quad \text{and} \quad \lim\limits_{s\to t^-}\widetilde{G}(x,t;\cdot,s) = \delta_x(\cdot).
    \end{equation*}
    Then for all $(y,s) \in M \times [0,T]$, $\widetilde{G}(\cdot,\cdot;y,s)$ is the heat kernel associated to the conjugate equation
    \begin{equation*}
        \left(\partial_t - \Delta^{h_t}_{x,t} - R_{h_t}\right)\widetilde{G}(\cdot,\cdot;y,s) = 0,\quad \text{and} \quad \lim\limits_{t\to s^+}\widetilde{G}(\cdot,t;y,s) = \delta_y(x).
    \end{equation*}
    Proposition 3.1 of \cite{bamler_ricci_2017} gives the following estimate: for any $n, A > 0$, there is a $C(n,A) < \infty$ such that the following holds: Let $(M, h(t)), t\in[0,T]$ be a complete Ricci flow satisfying
    \begin{equation}\label{eqn:BCW-estimate-heat-kernel-assumptions}
        |\text{Rm}(x,t)| \leq At^{-1},\quad \text{and} \quad \text{Vol}_{h_t}\left(B_{h_t}(x,\sqrt{t})\right) \geq A^{-1}t^{n/2}
    \end{equation}
    for all $(x,t)\in M\times(0,T]$. Then
    \begin{equation}\label{eqn:BCW-estimate}
        \widetilde{G}(x,t;y,s) \leq \frac{C}{t^{n/2}}\exp{\left(-\frac{d^2_s(x,y)}{Ct}\right)}
    \end{equation}
    for any $0 \leq s \leq t$. 
    
    Let $G(x,t;y,s)$ denote the heat kernel associated with the heat equation coupled with the Ricci Flow $h(t)$
    \begin{equation*}
        \left(\partial_t - \Delta^{h_t}_{x,t}\right)G(\cdot,\cdot;y,s) = 0,\quad \text{and}\quad \lim\limits_{t\to s^+}G(\cdot,t;y,s) = \delta_y(x).
    \end{equation*}
    Choosing the Ricci flow $h(t)$ above to be the one related to $g(t)$ via pullback by diffeomorphisms generated by $X$, that is, $h(t) = \chi_t^\ast g(t)$, then the heat kernel $K$ that we are interested is related to the heat kernel $G$ also by pullback by diffeomorphisms generated by $X$. We first show that under our assumptions, (\ref{eqn:BCW-estimate}) can be used to obtain an estimate for $G$. By our assumptions and Lemma \ref{lem:estimates-bootstrap}, since the scalar curvature $R_g$ can be written locally in an expression of order $\hdel g \ast \hdel g + \hdel^2 g$, we have the bound
    \begin{equation*}
        |R_g| \leq Ct^{-1+\alpha}
    \end{equation*}
    for some constant $C = C(n,\Lambda,k_1,k_2) < \infty$ and some small $\alpha = \alpha(p,\delta) > 0$. Then we have
    \begin{equation*}
        \left(\partial_t - \Delta^{h_t}_{x,t}\right)\widetilde{G} = R_{h_t}\widetilde{G} \geq -Ct^{\alpha-1}\widetilde{G}
    \end{equation*}
    which gives
    \begin{align*}
        \left(\partial_t - \Delta^{h_t}_{x,t}\right)(\exp{(\widetilde{C}t^\alpha)}\widetilde{G}) &\geq -Ct^{\alpha - 1}\exp{(\widetilde{C}t^\alpha)}\widetilde{G} + \alpha\widetilde{C}t^{\alpha-1}\exp{(\widetilde{C}t^\alpha)}\widetilde{G} \geq 0
    \end{align*}
    by choosing $\widetilde{C}$ appropriately. Since at time $t = 0$, $G = \widetilde{G}$, the maximum principle gives
    \begin{equation*}
        G \leq \exp{(\widetilde{C}t^\alpha)}\widetilde{G} \leq C\widetilde{G}
    \end{equation*}
    for some $C = C(n,\Lambda,k_1,k_2) > 0$. Note that our assumptions also ensures that the Ricci flow $h(t) = \chi_t^\ast g(t)$ satisfies the assumptions (\ref{eqn:BCW-estimate-heat-kernel-assumptions}) above. Finally, writing $K(x,t;y,s) = G(\chi_t^{-1}(x),t;\chi_s^{-1}(y),s)$ and after computations similar to that of Lemma 2.9 of \cite{burkhardt-guim_pointwise_2019}, we obtain the following estimate for $K$: there is a $C = C(n,\Lambda,T) > 0$ such that
    \begin{equation}\label{eqn:gradient-estimate-K-heat-kernel-estimate}
        K(x,t;y,s) \leq \frac{C}{t^{n/2}}\exp{\left(-\frac{d^2_h(x,y)}{Ct}\right)}
    \end{equation}
    for any $0 \leq s \leq t$.
    
    By (\ref{eqn:gradient-estimate-K-heat-kernel-estimate} and the co-area formula, we have on the right hand side of (\ref{eqn:gradient-estimate-invoke-heat-kernel})
    \begin{align}\label{eqn:gradient-estimate-Q-heat-kernel-estimate2}
        &\int_{M} K(x_0, T_1; y, 0)f(0)Q(y,0)d\mu_h(y) \nonumber \\
        &\quad \leq \int_{M} \frac{C_0}{T_1^{\frac{n}{2}}}\exp\left(-\frac{d^2_{h_0}(x_0, y)}{C_0 T_1}\right)\left(|\hdel g_0|^2 + \sigma^2\right)^{\frac{1}{2}}d\mu_h(y) \nonumber \\
        &= \int_0^{\infty} \frac{C_0}{T_1^{\frac{n}{2}}}\exp\left(-\frac{r^2}{C_0 T_1}\right)\left(\int_{\partial B(x_0, r)}\left(|\hdel g_0|^2 + \sigma^2\right)^{\frac{1}{2}}dS(y)\right)dr \nonumber \\
        &= \int_0^{\infty} \frac{C_0}{T_1^{\frac{n}{2}+1}}\exp\left(-\frac{r^2}{C_0 T_1}\right)\frac{2r}{C_0}\left(\int_{B(x_0, r)}\left(|\hdel g_0|^2 + \sigma^2\right)^{\frac{1}{2}}d\mu_h(y)\right)dr
    \end{align}
    where $dS$ is the surface area measure induced by $d\mu_h$ and $C_0$ above is the constant obtained from (\ref{eqn:gradient-estimate-K-heat-kernel-estimate}). For $p \geq 1$, assumption (iii) and H\"older's inequality gives
    \begin{equation*}
        \int_{B(x_0, r)} \left(|\hdel g_0|^2 + \sigma^2\right)^{\frac{1}{2}}d\mu_h(y) \leq CL_0r^{n-1+\frac{\delta}{p}} + 2\sigma r^n.
    \end{equation*}
    Substituting this into (\ref{eqn:gradient-estimate-Q-heat-kernel-estimate2}) we obtain,
    \begin{align*}
        &\int_0^{\infty} \frac{C_0}{T_1^{\frac{n}{2}+1}}\exp\left(-\frac{r^2}{C_0 T_1}\right)\frac{2r}{C_0}\left(\int_{B(x_0, r)}\left(|\hdel g_0|^2 + \sigma^2\right)^{\frac{1}{2}}d\mu_h(y)\right)dr \nonumber \\
        &\quad \leq \int_0^{\infty} \frac{4L_0}{T_1^{\frac{n}{2}+1}}\exp\left(-\frac{r^2}{C_0 T_1}\right)r^{n+\frac{\delta}{p}}dr + \int_0^{\infty} \frac{2}{T_1^{\frac{n}{2}+1}}\exp\left(-\frac{r^2}{C_0 T_1}\right)r^{n+1}\sigma dr \nonumber \\
        &\quad \leq \int_0^{\infty} \frac{4L_0}{T_1^{\frac{n}{2}+1}}\exp\left(-\frac{s^2}{C_0}\right)s^{n +\frac{\delta}{p}}T_1^{\frac{n}{2}+\frac{\delta}{2p}+\frac{1}{2}}ds + \frac{C_6\sigma}{T_1^{\frac{n}{2}+1}}T_1^{\frac{n}{2}+1} \nonumber \\
        &\quad \leq 4L_0T_1^{-\frac{1}{2}+\frac{\delta}{2p}}\int_0^\infty\exp\left(-\frac{s^2}{C_0}\right)s^{n+\frac{\delta}{p}}ds + C_6\sigma \nonumber \\
        &\quad\leq C_5T_1^{-\frac{1}{2}+\frac{\delta}{2p}} + C_6\sigma
    \end{align*}
    for some positive constants $C_5, C_6$ depending on $n, \Lambda, L_0, k_1$ that may be changing line by line and where we used the change of variables $r = s\sqrt{T_1}$ in the second inequality above. So we have
    \begin{align}\label{eqn:gradient-estimate-Q-heat-kernel-estimate4}
        e^{-AT_1^{\frac{\delta}{p}}-A\sigma^2T_1}B_1T_1^{-\frac{1}{2}+\frac{\delta}{2p}} &\leq C_5T_1^{-\frac{1}{2}+\frac{\delta}{2p}} + C_6\sigma + C_3 T_1\nonumber \\
        &\leq (C_5 + C_6 + C_3)T_1^{-\frac{1}{2}+\frac{\delta}{2p}}
    \end{align}
    where the last inequality is again using the fact that $T_1 \leq 1$ and also that $\sigma < 1$. Clearing the denominators of both sides and choosing $B_1 = e(C_5 + C_6 + C_3)$, then taking logarithm of both sides, we obtain
    \begin{equation*}
        T_1 \geq \left(\frac{1}{A}-\sigma^2\right)^{\frac{p}{\delta}}.
    \end{equation*}
    Tracing through (\ref{eqn:gradient-estimate-box-Q3}), we find that with our choice of $B_1$,
    \begin{equation*}
        A \geq \max\left\{ C_2\sigma^2, \frac{C_2e(C_5+C_6+C_3)p}{\delta}\right\}
    \end{equation*}
    for small $\sigma$. So letting $\sigma \to 0$, we can conclude that $T_1 > 0$ uniformly and we are done.
\end{proof}

Note that now that the $C^1$ estimate has been established, Lemma \ref{lem:estimates-bootstrap} gives us all higher order estimates.

\subsection{Preservation Along the Flow}

We now show that under the interior estimates, the $L^\infty$ condition is preserved along the flow.

\begin{lemma}\label{lem:Linfty-preservation}
    Suppose $g(t)$ is a smooth solution to the Ricci-DeTurck $h$-flow on $M \times [0,T_1]$ for some smooth background metric $h$ satisfying (\ref{eqn:h-remark-curvature-estimates}) as in Remark \ref{rmk:estimates-h-remark} so that
    \begin{enumerate}[(i)]
        \item $\Lambda_0^{-1}h \leq g(0) \leq \Lambda_0 h$ on $M$ for some $\Lambda_0 > 1$;
        \item $g(t)$ satisfies (\ref{eqn:bootstrap-higher-order-interior-estimate}) on $(0, T_1]$,
    \end{enumerate}
    then there is a $\Lambda = \Lambda(n, \Lambda_0,B_1,B_2)$ such that
    \begin{equation*}
        \Lambda^{-1}h \leq g(t) \leq \Lambda h
    \end{equation*}
    on $M \times (0,T_1]$ where the constants $B_1,B_2$ are obtained from (\ref{eqn:bootstrap-higher-order-interior-estimate}).
\end{lemma}

\begin{proof}
    In our case, this estimate is made easier by the fact that the interior estimates we derived are not scale invariant, in particular, they are integrable in $t$. By (\ref{eqn:prelim-ricci-deturck-flow}) and (\ref{eqn:bootstrap-higher-order-interior-estimate}), for any $X \in TM$, we have
    \begin{equation*}
        \left|\frac{\partial}{\partial t} g_t(X,X)\right| \leq C_4t^{-1+\frac{\delta}{2p}}g_t(X,X)
    \end{equation*}
    where $C_4$ depends on $n, \Lambda_0, B_1, B_2$. Then taking logarithm and integrating from 0 to $t$ gives the desired result.
\end{proof}

We now show that the Morrey condition is preserved along the flow.

\begin{lemma}\label{lem:morrey-preservation}
    Suppose $g(t)$ is a smooth solution to the Ricci-DeTurck $h$-flow on $M \times [0, T]$ for some smooth background metric $h$ satisfying (\ref{eqn:h-remark-curvature-estimates}) as in Remark \ref{rmk:estimates-h-remark} so that:
    \begin{enumerate}[(i)]
        \item $\Lambda^{-1}h \leq g(t) \leq \Lambda h$ on $M\times[0,T]$ for some $\Lambda > 1$;
        \item for $p \geq 1, \delta > 0$, there is a $L_0 > 0$ such that for all $x_0 \in M$ and $0 < r < r_0$,
        \begin{equation*}
            \int_{B(x_0,r)} |\hdel g(x,0)|^p d\mu_h(x) \leq L_0 r^{n-p+\delta}.
        \end{equation*}
        Then there is a $L(n,\Lambda_0,L_0,p,\delta) > 0, T_1 > 0$ such that for all $x_0 \in M$ and $0 < r < r_0$
        \begin{equation*}
            \int_{B(x_0,r)} |\hdel g(t)|^p d\mu_h \leq L r^{n-p+\delta}
        \end{equation*}
        for all $t \in (0, T_1]$.
    \end{enumerate}
\end{lemma}

\begin{proof}
    Fix $t \in [0, T_1]$. We split the proof into two cases:
    \begin{enumerate}[(1)]
        \item $r < \sqrt{t}$
        \item $r \geq \sqrt{t}$
    \end{enumerate}

    Let $\eta$ be a smooth cut-off function on $B(x_0, r)$ satisfying (a) - (e) as in the proof of Lemma \ref{lem:estimates-bootstrap} and let $f, Q, C_3$ be as in (\ref{eqn:gradient-estimate-box-Q3}) of Lemma \ref{lem:gradient-estimate} above. That is,
    \begin{equation*}
        Q := \left(|\hdel g|^2 + \sigma^2\right)^\frac{1}{2}
    \end{equation*}
    for $\sigma \in (0,1)$, and $f(t) := \exp\left(-A(t^\frac{\delta}{p}+\sigma^2t)\right)$. Then by (\ref{eqn:gradient-estimate-box-Q3}),
    \begin{align*}
        &\int_{B(x_0,r)} \left(f(t)Q(x,t)-C_3t\right)d\mu_h(x) \nonumber \\
        &\quad\leq \int_M \left(f(t)Q(x,t)-C_3t\right)\eta(x)d\mu_h(x) \nonumber \\
        &\quad\leq \int_M \int_M K(x,t;y,0)f(0)Q(y,0)\eta(x)d\mu_h(y)d\mu_h(x) \nonumber \\
        &\quad\leq \int_M \left(C_5t^{-\frac{1}{2}+\frac{\delta}{2p}}+C_6\sigma\right)\eta(x)d\mu_h(x) \nonumber \\
        &\quad\leq C_5t^{-\frac{1}{2}+\frac{\delta}{2p}}r^n + C_6\sigma r^n \nonumber \\
        &\quad\leq C_5r^{n-1+\frac{\delta}{p}} + C_6\sigma r^n
    \end{align*}
    where we are using the computation and the constants $C_5$ and $C_6$ from (\ref{eqn:gradient-estimate-Q-heat-kernel-estimate2}) in the third inequality above, and the assumption that $r < \sqrt{t}$ in the last inequality above. Note in particular that $C_5$ above depends on $n, \Lambda_0, L_0$. Then since $f(t) \geq f(1) > 0$ and we can take $t \leq T_1 \leq 1$, we obtain
    \begin{equation*}
        \int_{B(x_0,r)} Q(x,t)d\mu_h(x) \leq C_5r^{n-1+\frac{\delta}{p}} + C_6\sigma r^n + C_3r^n.
    \end{equation*}
    Adjusting the constants if necessary and letting $\sigma \to 0$, we obtain the desired result for $r < \sqrt{t}$ and for $p = 1$.

    For $p > 1$, by (\ref{lem:gradient-estimate}), by $r < \sqrt{t}$ we have $t^{-\frac{1}{2}} < r^{-1}$ and so
    \begin{align*}
        \int_{B(x_0,r)}|\hdel g|^pd\mu_h &\leq \int_{B(x_0,r)}|\hdel g||\hdel g|^{p-1}d\mu_h \nonumber \\
        &\leq \left(\int_{B(x_0,r)}|\hdel g|d\mu_h\right)\sup\limits_{x\in M}|\hdel g|^{p-1} \nonumber \\
        &\leq LB_1 r^{n-1+\delta}\left(t^{-\frac{1}{2}\left(1-\frac{\delta}{p}\right)}\right)^{p-1} \nonumber \\
        &\leq L r^{n-p+\delta}
    \end{align*}
    for $0 < r < \min\{\sqrt{t}, r_0\} \leq 1$.
    
    For the second case, we proceed by an energy argument. For this, we introduce a cut-off function. Let $\phi:[0,\infty)\to\mathbb{R}$ be a smooth function such that $\phi \equiv 1$ on $[0,1]$, $\phi \equiv 0$ on $[2,\infty)$ and satisfies $|\phi'|\leq 10^4, \phi'' \geq -10^4\phi$. Fix $x_0 \in M$. Obtained from \cite{tam_exhaustion_2010}, let $\rho(\cdot)$ be uniformly equivalent to $d_h(x_0, \cdot)$ and satisfying
    \begin{equation*}
        |\hdel \rho|^2 + |\hdel^2 \rho| \leq C
    \end{equation*}
    for some $C > 0$. Then define $\Phi(x) := \phi^m(\rho(x)r^{-1})$ for some $m \in \mathbb{N}$ to be determined. That is, $\Phi \equiv 1$ on $B(x_0, r)$ and vanishes outside $B(x_0, 2r)$. 

    By the first inequality in (\ref{eqn:gradient-estimate-partialt-Qsquared}), we have that
    \begin{equation*}
        \frac{\partial}{\partial t}Q  \leq g^{ij}\hdel_i\hdel_j Q + C_2Q^3 + C_3.
    \end{equation*}
    Then $Q^p = \left(|\hdel g|^2 + \sigma^2\right)^\frac{p}{2}$ satisfies
    \begin{align*}
        \frac{\partial}{\partial t} Q^p &\leq g^{ij}\hdel_i\hdel_j Q^p + C_2B_1t^{-1+\frac{\delta}{p}}Q^p + (pC_3 + C_2\sigma^2)Q^p.
    \end{align*}
    Letting $l(t) = \exp\left(-A(t^\frac{\delta}{p}-t)\right)$ for some constant $A > 0$ to be determined, then we have
    \begin{equation}\label{eqn:morrey-preservation-partialt-lQp}
        \frac{\partial}{\partial t} (lQ^p) = l\partial_t Q^p + lQ^p\left(-\frac{A\delta}{p}t^{-1+\frac{\delta}{p}} - A\right) \leq g^{ij}\hdel_i\hdel_j(lQ^p)
    \end{equation}
    for $A$ large enough.
    
    We define the energy quantity
    \begin{equation*}
        E(t) = \int_{M} l(t)Q^p(x,t)\Phi(x)d\mu_h(x).
    \end{equation*}

    By (\ref{eqn:morrey-preservation-partialt-lQp}) and integrating by parts,
    \begin{align}\label{eqn:morrey-preservation-energy-derivative1}
        \frac{\partial}{\partial t} E(t) &\leq C_4\int_M lQ^p|\hdel^2\Phi| d\mu_h + C_4\int_M lQ^p|\hdel g||\hdel\Phi|d\mu_h \nonumber \\
        &\qquad + C_4\int_M (lQ^p\Phi)\left(|\hdel g|^2 + |\hdel^2 g|\right)d\mu_h =: \text{I} + \text{II} + \text{III}.
    \end{align}
    where $C_4$ is a constant depending on $n$ and $\Lambda$.
    
    We now estimate each term above. We have
    \begin{equation}\label{eqn:morrey-preservation-I-estimate1}
        \text{I} \leq C_4 r^{-2}\int_MlQ^p\Phi^{1-\frac{1}{m}}d\mu_h = C_4r^{-2}\int_M (lQ^p\Phi)^{1-\frac{1}{m}}(lQ^p)^\frac{1}{m}d\mu_h.
    \end{equation}
    We now apply the gradient estimate in $B(x_0,2r)$ on the $(lQ^p)^\frac{1}{m}$ term above. Since both $l, \sigma$ are uniformly bounded from above, by (\ref{eqn:interior-gradient-estimate}), we have
    \begin{equation*}
        (lQ^p)^\frac{1}{m} \leq C_5t^{\frac{p}{2m}\left(-1+\frac{\delta}{p}\right)}.
    \end{equation*}
    Inserting this into (\ref{eqn:morrey-preservation-I-estimate1}) and by H\"older's inequality, we have
    \begin{align}\label{eqn:morrey-preservation-I-estimate2}
        C_4r^{-2}\int_M(lQ^p\Phi)^{1-\frac{1}{m}}(lQ^p)^\frac{1}{m}d\mu_h &\leq C_5r^{-2}t^{\frac{p}{2m}\left(-1+\frac{\delta}{p}\right)}E(t)^{1-\frac{1}{m}}\text{Vol}({B(x_0,2r)})^\frac{1}{m} \nonumber \\
        &\leq C_5E(t)^{1-\frac{1}{m}}r^{\frac{n}{m}-2}t^{\frac{p}{2m}\left(-1+\frac{\delta}{p}\right)}.
    \end{align}
    Similarly for II, we have
    \begin{align}\label{eqn:morrey-preservation-II-estimate}
        \text{II} &\leq C_4E(t)^{1-\frac{1}{m}}r^{\frac{n}{m}-1}t^{-\frac{1}{2}+\frac{\delta}{2p}+\frac{p}{2m}\left(-1+\frac{\delta}{p}\right)}.
    \end{align}
    For III, we use (\ref{eqn:bootstrap-higher-order-interior-estimate}) and similarly,
    \begin{align}\label{eqn:morrey-preservation-III-estimate}
        \text{III} &\leq C_4E(t)^{1-\frac{1}{m}}r^\frac{n}{m}\left(t^{-1+\frac{\delta}{p}+\frac{p}{2m}\left(-1+\frac{\delta}{p}\right)} + t^{-1+\frac{\delta}{2p}+\frac{p}{2m}\left(-1+\frac{\delta}{p}\right)}\right).
    \end{align}

    Combining (\ref{eqn:morrey-preservation-I-estimate2}), (\ref{eqn:morrey-preservation-II-estimate}), (\ref{eqn:morrey-preservation-III-estimate}) into (\ref{eqn:morrey-preservation-energy-derivative1}), we obtain
    \begin{align*}
        \frac{\partial}{\partial t} E(t) &\leq C_6E(t)^{1-\frac{1}{m}}\left(r^{\frac{n}{m}-2}t^{\frac{p}{2m}\left(-1+\frac{\delta}{p}\right)} + r^{\frac{n}{m}-1}t^{-\frac{1}{2}+\frac{\delta}{2p}+\frac{p}{2m}\left(-1+\frac{\delta}{p}\right)} \right. \nonumber \\
        &\qquad\qquad\qquad\quad \left. + r^{\frac{n}{m}}\left(t^{-1+\frac{\delta}{p}+\frac{p}{2m}\left(-1+\frac{\delta}{p}\right)} + t^{-1+\frac{\delta}{2p}+\frac{p}{2m}\left(-1+\frac{\delta}{p}\right)}\right)\right).
    \end{align*}
    Hence, we have
    \begin{align*}
        \frac{\partial}{\partial t}\left(E(t)\right)^{\frac{1}{m}} &= \frac{1}{m}E(t)^{\frac{1}{m}-1}\frac{\partial}{\partial t}E(t) \nonumber \\
        &\leq C_6\left(r^{\frac{n}{m}-2}t^{\frac{p}{2m}\left(-1+\frac{\delta}{p}\right)} + r^{\frac{n}{m}-1}t^{-\frac{1}{2}+\frac{\delta}{2p}+\frac{p}{2m}\left(-1+\frac{\delta}{p}\right)} \right. \nonumber \\
        &\qquad\quad \left. + r^{\frac{n}{m}}\left(t^{-1+\frac{\delta}{p}+\frac{p}{2m}\left(-1+\frac{\delta}{p}\right)} + t^{-1+\frac{\delta}{2p}+\frac{p}{2m}\left(-1+\frac{\delta}{p}\right)}\right)\right).
    \end{align*}
    Then integrating from $0$ to $t$, we obtain
    \begin{align*}
        \left(E(t)\right)^{\frac{1}{m}} &\leq \left(E(0)\right)^\frac{1}{m} + C_6\left(r^{\frac{n}{m}-2}t^{\frac{1}{2}\left(2+\frac{p}{m}\left(-1+\frac{\delta}{p}\right)\right)} + r^{\frac{n}{m}-1}t^{\frac{1}{2}\left(1+\frac{\delta}{p}+\frac{p}{m}\left(-1+\frac{\delta}{p}\right)\right)} \right. \nonumber \\
        &\qquad\qquad\qquad\qquad \left. + r^{\frac{n}{m}}\left(t^{\frac{1}{2}\left(\frac{2\delta}{p}+\frac{p}{m}\left(-1+\frac{\delta}{p}\right)\right)} + t^{\frac{1}{2}\left(\frac{\delta}{p}+\frac{p}{m}\left(-1+\frac{\delta}{p}\right)\right)}\right)\right). \nonumber \\
    \end{align*}
    We find that when
    \begin{equation*}
        m > \max\left\{\frac{p-\delta}{2},\frac{p-\delta}{1+\frac{\delta}{p}},\frac{p(p-\delta)}{2\delta},\frac{p(p-\delta)}{\delta}\right\},
    \end{equation*}
    all the half-powers of $t$ above are positive, and so by applying the assumption that $\sqrt{t}\leq r$, we have
    \begin{align*}
        \left(E(t)\right)^{\frac{1}{m}} &\leq \left(E(0)\right)^\frac{1}{m} + C_6\left(r^{\frac{1}{m}\left(n-p+\delta\right)} + r^{\frac{1}{m}\left(n-p+\delta\right)+\frac{\delta}{p}} + r^{\frac{1}{m}\left(n-p+\delta\right)+\frac{2\delta}{p}}\right) \nonumber \\
        &\leq \left(E(0)\right)^\frac{1}{m} + C_6r^{\frac{1}{m}\left(n-p+\delta\right)}
    \end{align*}
    choosing the dominant power of $r$ for the range $0 < r < r_0 \leq 1$ and enlarging $C_6$ if necessary. We now apply assumption (ii) above to obtain
    \begin{align*}
        \left(E(t)\right)^{\frac{1}{m}} &\leq (L_0 + C_6)r^{\frac{1}{m}\left(n-p+\delta\right)} + C_4\sigma^\frac{p}{m}r^{\frac{n}{m}}.
    \end{align*}
    So taking $m$-th powers we obtain
    \begin{equation*}
        \int_{B(x_0, r)} l(t)\left(|\hdel g(x,t)|^2 + \sigma^2\right)^{\frac{1}{2}}d\mu_h(x) \leq E(t) \leq (C_6+L_0)r^{n-p+\delta} + C_4^m\sigma^p r^n.
    \end{equation*}
    Then taking $\sigma \to 0$ yields the desired result.
    \begin{comment}
    For $p > 1$, we consider the quantity
    \begin{equation}
        Q^p = \left(|\hdel g|^2 + \sigma^2\right)^\frac{p}{2}
    \end{equation}
    which satisfies the inequality
    \begin{align}
        \frac{\partial}{\partial t} Q^p &\leq g^{ij}\hdel_i\hdel_j Q^p - p(p-1)Q^{p-2}|\hdel Q|^2 + C_2Q^{p+2} + C_3Q^{p-1} \nonumber \\
        &\leq g^{ij}\hdel_i\hdel_j Q^p - p(p-1)Q^{p-2}|\hdel Q|^2 + C_2B_1t^{-1+\frac{\delta}{p}}Q^p + C_3Q^p \nonumber \\
        &\leq g^{ij}\hdel_i\hdel_j Q^p + C_2B_1t^{-1+\frac{\delta}{p}}Q^p + C_3Q^p.
    \end{align}
    Hence
    \begin{equation}
        \frac{\partial}{\partial t} (l(t)Q^p) \leq g^{ij}\hdel_i\hdel_j(lQ^p)
    \end{equation}
    for $l(t) = \exp\left(-A(t^\frac{\delta}{p}+t)\right)$ for $A > 0$ large enough. Let $\Phi$ be the same cut-off function as before and we define
    \begin{equation}
        E_p(t) = \int_M lQ^p\Phi d\mu_h.
    \end{equation}
    Then we have
    \begin{align}
        \frac{\partial}{\partial t} E_p(t) &\leq \int_M g^{ij}\hdel_i\hdel_j(lQ^p)\Phi d\mu_h \nonumber \\
        &= \int_M lQ^p\hdel_i\hdel_j(g^{ij}\Phi)d\mu_h \nonumber \\
        &\leq C_4r^{-2}\int_M lQ^p\Phi^{1-\frac{1}{m}}d\mu_h + C_4\int_M lQ^p|\hdel g||\hdel\Phi|d\mu_h \nonumber \\
        &\quad + C_4\int_M lQ^p\left(|\hdel g|^2 + |\hdel^2g|\right)d\mu_h.
    \end{align}
    Then, performing much of the same computations as in the case for $p=1$ above, we arrive at
    \begin{align}
        \frac{\partial}{\partial t} (E_p(t))^\frac{1}{m} &\leq C_4r^\frac{n}{m}\left(t^{-1-\frac{p}{2m}+\frac{\delta}{p}+\frac{\delta}{2m}} + t^{-1-\frac{p}{2m}+\frac{\delta}{2p}+\frac{\delta}{2m}} + r^{-2}t^{\frac{p}{2m}(-1+\frac{\delta}{p})} \right. \nonumber \\
        &\qquad\qquad\quad \left. + r^{-1}t^{-\frac{1}{2}-\frac{p}{2m}+\frac{\delta}{2p}+\frac{\delta}{2m}}\right).
    \end{align}
    In particular, we apply assumption (ii) for $p > 1$. Integrating from $0$ to $t$ and applying the assumption $\sqrt{t} \leq r$, we obtain
    \begin{align}
        E_p(t)^\frac{1}{m} &\leq E_p(0)^\frac{1}{m} + C_4r^{\frac{1}{m}\left(n-p+\delta\right)} \nonumber \\
        &\leq (C_M + C_4)r^{\frac{1}{m}(n-p+\delta)} + \sigma^\frac{1}{m}r^n
    \end{align}
    for $m$ large enough. Finally, taking $m$-th powers and letting $\sigma \to 0$, we obtain the result for $p > 1$.
    \end{comment}
\end{proof}

\section{Proof of Main Theorem}\label{sec:proof-of-main-theorem}

We are now in a position to prove the main result.

\begin{proof}[Proof of Theorem \ref{thm:intro-main-theorem}]
    Since $g_0$ satisfies (i), by Theorem A.1 of \cite{lamm_ricci_2021} (which is a modification of Shi's classical existence theory in \cite{shi_deforming_1989}), there is a short-time solution to the Ricci-DeTurck $h$-flow $g(t)$ on $M \times [0,S]$ for some $S > 0$ such that $g(0) = g_0$ and there exist constants $b_m(g_0, h, S) < \infty$ such that $\sup\limits_M |\hdel^m g(\cdot, t)| \leq b_m < \infty$ for all $t \in [0, S]$. Note that since $|\text{Rm}(h)|\leq 1$, we can take $h$ to satisfy (\ref{eqn:h-remark-curvature-estimates}) as in Remark \ref{rmk:estimates-h-remark}.

    By the additional assumption (ii) and Lemmas \ref{lem:gradient-estimate}, \ref{lem:estimates-bootstrap}, \ref{lem:Linfty-preservation}, and \ref{lem:morrey-preservation}, we then see that there is a $T(n, \Lambda_0, L_0, p, \delta) > 0$ with $T \leq S$ such that the solution $g(t)$ obtained above satisfies also (a) - (c) for all $t \in (0, T]$.
\end{proof}

\section{Applications}\label{sec:applications}

In this chapter, we use the existence theory established above to study a number of applications.

\subsection{Ricci-DeTurck Flow from Singular Metric}

We will be interested in the case where $M^n$ is either compact or complete non-compact and with an explicit singularity contained in a compact region. Before the proof of Theorem \ref{thm:intro-application-1-statement}, we first obtain smooth approximations of a H\"older continuous metric $g_0$ by a mollification scheme and show that such approximations satisfy a number of desired properties.

\begin{lemma}\label{lem:mollification-scheme}
    Suppose $g_0 \in C^\alpha(M)$ for some $\alpha \in (0,1)$ and $g_0$ is smooth away from a compact set $\Sigma$. Moreover, suppose $g_0$ satisfies:
    \begin{enumerate}[(i)]
        \item there is a $\Lambda_0 > 1$ such that on $M$,
        \begin{equation*}
            \Lambda_0^{-1}h \leq g_0 \leq \Lambda_0 h;
        \end{equation*}
        \item for $p \geq 1, \delta > 0, r_0 > 0$, there is a $L_0 > 0$ such that for all $x_0 \in M$ and $0 < r < r_0$,
        \begin{equation*}
            \fint_{B(x_0,r)} |\hdel g_0|^p d\mu_h \leq L_0 r^{-p+\delta}.
        \end{equation*}
    \end{enumerate}
    Then there is a sequence of smooth metrics $g_{i,0}$ on $M$ such that $g_{i,0} = g_0$ outside $\Sigma(i^{-1})$, and satisfies:
    \begin{enumerate}[(a)]
        \item for $i$ sufficiently large,
        \begin{equation*}
            (2\Lambda_0)^{-1}h \leq g_{i,0} \leq 2\Lambda_0 h
        \end{equation*}
        on $M$;
        \item each $g_{i,0}$ also satisfies (ii) above;
        \item $g_{i,0} \to g_0$ in $C^0$ on $M$ as $i \to \infty$;
        \item $g_{i,0} \to g_0$ locally uniformly smoothly outside $\Sigma$ as $i\to\infty$;
        \item $g_{i,0} \to g_0$ in $W^{1,p}_\text{loc}(M)$.
    \end{enumerate}
\end{lemma}

\begin{proof}
    Since $\Sigma$ is compact, there is a point $p \in M$ and $R > 0$ such that $\Sigma \subset B(p,R)$. By \cite{tam_exhaustion_2010} and $|\text{Rm}(h)|\leq 1$, there is $\rho \in C^\infty_\text{loc}(M)$ such that $|\hdel \rho|^2 + |\hdel^2 \rho| \leq 1$ and
    \begin{equation*}
        C_n^{-1}(d_h(\cdot, p)+1)\leq \rho(\cdot) \leq C_n(d_h(\cdot,p) + 1)
    \end{equation*}
    for some dimensional constant $C_n$. Let $\phi$ be a smooth function on $[0,+\infty)$ such that $\phi \equiv 1$ on $[0,1]$, $\phi \equiv 0$ on $[2,+\infty)$ and $0\leq-\phi'\leq 10$. Let $K$ denote the support of the function $\phi(\frac{\rho(\cdot, p)}{R})$. Note that $K$ is compact. We now cover $K$ by finitely many open coordinate charts $\{U_k\}_{k=1}^N$. Also let $U_0 = M \setminus K$. Let $\varphi_k$ be a partition of unity subordinate to $U_0 \cup \bigcup\limits_{k=1}^N U_k$. Then we can decompose the metric on each chart by $g_0^k = \varphi_k g_0$. For $k = 1,\dots,N$, let $\eta$ be the standard mollifier with $\int_{U_k} \eta(y)dy = 1$ and define
    \begin{equation*}
        g_{i,0}^k(x) := \int_{U_k} g_0^k(x-i^{-1}y)\eta(y)dy.
    \end{equation*}
    Then we define
    \begin{equation*}
        g_{i,0}(x) := \phi(R^{-1}\rho)\sum\limits_{k=1}^N g_{i,0}^k + (1-\phi(R^{-1}\rho))g_0.
    \end{equation*}
    Outside $K$, $g_{i,0}$ coincides with $g_0$, while on $K$, $g_{i,0}$ is a mollification of $g_0$. We now check that the above properties are satisfied. For (a) and (b) above, it suffices to show that they are satisfied for $x \in K$.
    
    Suppose $x \in U_k$ for some $k = 1,\dots,N$. Note that since $h$ is smooth, for all $\varepsilon > 0$, $y\in M$, $|h(x-i^{-1}y) - h(x)| \leq \varepsilon$ for $i$ large enough, that is,
    \begin{equation*}
        h_{ij}(x) - \varepsilon\delta_{ij} \leq h_{ij}(x-i^{-1}y) \leq \varepsilon\delta_{ij} + h_{ij}(x)
    \end{equation*}
    where $\delta_{ij}$ is the Kronecker delta. Then by (i), (a) follows for $i$ sufficiently large.
    
    Let $x_0 \in M$. It suffices to show that the Morrey-type condition is preserved on each $U_k$, $k = 1,\dots,N$. For fixed $y$ with $|y| \leq 1$, note that $z:= x-i^{-1}y$ is a translation with determinant of Jacobian uniformly bounded from above and below. We use the fact that mollification and differentiation commute, i.e. $\partial g_i(x) = (\eta_i \ast \partial g)(x)$, then by Minkowski's integral inequality,
    \begin{align*}
        &\left(\int_{B(x_0,r)}|\partial g_{i,0}^k(x)|^pd\mu_h(x)\right)^\frac{1}{p} \nonumber \\
        &\quad= \left(\int_{B(x_0,r)}\left|\int_{U_k} |\partial g_0^k(x-i^{-1}y)\eta(y)dy\right|^{p}d\mu_h(x)\right)^\frac{1}{p} \nonumber \\
        &\quad\leq \int_{U_k} \left(\int_{B(x_0,r)}|\partial g(x-i^{-1}y)\eta(y)|^pdx\right)^\frac{1}{p}dy \nonumber \\
        &\quad\leq \int_{U_k}\eta(y)dy\left(\int_{B(x_0,r)} |\partial g_0^k(x-i^{-1}y)|^p d\mu_h(x)\right)^\frac{1}{p} \nonumber \\
        &\quad\leq \left(C\int_{B(x_0 + i^{-1}y, r)} |\partial g_0^k(z)|^p dz\right)^\frac{1}{p} \nonumber \\
        &\quad\leq \left(CL_0 r^{n-p+\delta}\right)^\frac{1}{p}.
    \end{align*}
    So (b) is satisfied.
    
    Let $\varepsilon > 0$ be given. If $x \in M\setminus K$, then (c) follows o. If $x \in K$, then $x \in U_k$ for some $k = 1,\dots,N$. By the H\"older continuity of $g_0$, for $i$ sufficiently large, $|g_0(x-i^{-1}y) - g_0(x)| < C|i^{-1}y|^\alpha <\varepsilon$ for any $x, y \in M$. Hence by $\int_{U_k} \eta(y)dy = 1$,
    \begin{align*}
        |g_{i,0}^k(x) - g_0(x)| &= \left|\int_{U_k} g_0^k(x-i^{-1}y)\eta(y)dy - g_0(x)\right| \nonumber \\
        &\leq \int_{U_k} | g_0^k(x-i^{-1}y) - g_0(x)|\eta(y)dy \nonumber \\
        &\leq \int_{U_k} |i^{-1}y|^\alpha \eta(y)dy \nonumber \\
        &\leq \int_{U_k} \varepsilon \eta(y)dy \nonumber \\
        &= \varepsilon.
    \end{align*}
    
    For (d), for $x \in \Omega \subset M \setminus \Sigma$, it is standard that $\partial^\ell g_{i,0}^k$ is uniformly bounded for $\ell \geq 1$. So each $g_{i,0}^\ell \in C^\infty_\text{loc}$. So after a covering argument we can conclude that $g_{i,0}$ converges to $g_0$ in $C^\infty_\text{loc}$ as $i\to\infty$.
    
    Finally for (e), since $g_0$ satisfies the Morrey condition with $n - p + \delta > 0$, when $U \subset M$ is compact, a standard covering argument shows that $g_0 \in W^{1,p}(U)$. Then the standard mollification above gives $g_{i,0} \to g_0$ in $W^{1,p}(U)$ as required.
\end{proof}

There are many other similar approximation schemes, see for example \cite{lee_positive_2013}, \cite{shi_scalar_2016}, \cite{lee_continuous_2021}, and \cite{grant_positive_2014} and references therein.

We are now ready to prove Theorem \ref{thm:intro-application-1-statement}.

\begin{proof}[Proof of Theorem \ref{thm:intro-application-1-statement}]
    Note that since $|\text{Rm}|\leq 1$, we can take $h$ to satisfy (\ref{eqn:h-remark-curvature-estimates}) as in Remark \ref{rmk:estimates-h-remark}. By the embedding theorem of Morrey (see for example ``Morrey's Lemma'' (1.3) of \cite{adams_morrey_2015}), assumption (iii) implies that $g_0$ is H\"older continuous with exponent $\delta/p < 1$. Then we take $g_{i,0}$ as in Lemma \ref{lem:mollification-scheme} above.

    By properties (a), (b) in Lemma \ref{lem:mollification-scheme} above, we know that for $i$ sufficiently large, that
    \begin{equation}\label{eqn:rough-initial-gi0-Linfty}
        (2\Lambda_0)^{-1}h \leq g_{i,0} \leq 2\Lambda_0 h,
    \end{equation}
    and
    \begin{equation}\label{eqn:rough-initial-gi0-Morrey}
        \fint_{B(x_0, r)} |\hdel g_{i,0}|^p d\mu_h \leq L r^{-p+\delta}
    \end{equation}
    for some $L = L(n, \Lambda_0, L_0, p, \delta, r_0) > 0$.

    Then since each $g_{i,0}$ is smooth and satisfies (\ref{eqn:rough-initial-gi0-Linfty}) and (\ref{eqn:rough-initial-gi0-Morrey}) above for $i$ large enough, by Theorem 1.1, we obtain constants $T_i = T_i(n,h,\Lambda_0,p,\delta) > 0$, $\Lambda(n, \Lambda_0, L_0, p, \delta, r_0) > 1$, $B_k(n, k, \Lambda_0, L_0, p, \delta, R) > 0$ and solutions $g_i(t)$ to the Ricci-DeTurck $h$-flow on $M \times [0,T_i]$ such that $g_i(0) = g_{i,0}$ satisfying
    \begin{enumerate}[(i')]
        \item $\Lambda^{-1}h \leq g_i(t) \leq \Lambda h$ on $M \times [0,T_i]$;
        \item $\fint_{B(x_0,r)} |\hdel g_i(t)|^p d\mu_h \leq Lr^{-p+\delta}$ for all $x_0 \in M$, $t \in [0,T_i]$;
        \item $\sup\limits_{B(x_0, R)}|\hdel^k g_i(t)| \leq \frac{B_k}{t^{\frac{1}{2}(m-\frac{\delta}{p})}}$ on $M \times [0,T_i]$.
    \end{enumerate}
    Since the above hold up to $t = T_i$, by a continuity argument, we have that $T_i$ is uniformly bounded from below by some $S(n,\Lambda_0) > 0$ independent of $i$.

    Then restricting $g_i(t)$ on $M \times [0,S]$, Lemmas \ref{lem:gradient-estimate} and \ref{lem:estimates-bootstrap} or (iii') above show that $g_i(t)$ is uniformly $C_\text{loc}^k(M)$ bounded for any $[a,b] \subset (0, S]$, and uniformly $L^\infty$ with constant $\Lambda$ on $[0,S]$. So by the Arzel\`a--Ascoli theorem and taking a subsequence, we obtain smooth $g(t) = \lim\limits_{i\to\infty}g_i(t)$ on $M \times [0,S]$ with $g(t)$ satisfying (a) - (c) above.

    (d) follows by the proof of Theorem 5.2 of \cite{simon_deformation_2002}. Finally, by applying Proposition \ref{prop:time-zero-smoothness}, we obtain (e). As remarked above, our $L^\infty$ condition differs from condition (i) in Proposition \ref{prop:time-zero-smoothness}. However, our assumptions allow us to find a fixed $t_0$ for which $g(t)$ satisfies condition (i) with respect to the metric $g(t_0)$ (in place of $h$) in Proposition \ref{prop:time-zero-smoothness}. We then apply Proposition \ref{prop:time-zero-smoothness} with respect to the metric $g(t_0)$ and use the fact that $g(t_0)$ is uniformly $C^k$ close to $h$ to obtain (e) with respect to $h$.
\end{proof}

\subsection{Distributional Scalar Curvature}

Using the rough existence theory, we study some applications related to distributional scalar curvature. In \cite{lee_positive_2015}, Lee--LeFloch introduced a notion of distributional scalar curvature that is defined for metrics with low regularity. In particular, for any $g \in L^\infty_{\text{loc}}\cap W^{1,2}_{\text{loc}}$ with locally bounded inverse $g^{-1}\in L^\infty_{\text{loc}}$, they define the scalar curvature distribution $R_g$ by
\begin{equation}\label{eqn:distributional-scalar-defn}
    \left\langle R_g, u\right\rangle := \int_M \left(-V\cdot\hdel\left(u\frac{d\mu_g}{d\mu_h}\right)+Fu\frac{d\mu_g}{d\mu_h}\right)d\mu_h
\end{equation}
for every compactly supported smooth test function $u:M\to\mathbb{R}$ and where
\begin{align*}
    &\Gamma^k_{ij} := \frac{1}{2}g^{k\ell}\left(\hdel_ig_{j\ell}+\hdel_jg_{i\ell}-\hdel_{\ell}g_{ij}\right) \\
    &V^k := g^{ij}\Gamma^{k}_{ij} - g^{ik}\Gamma^{j}_{ji} \\
    &F := \tilde{R} - \hdel_kg^{ij}\Gamma^k_{ij} + \hdel_kg^{ik}\Gamma^{j}_{ji} + g^{ij}\left(\Gamma^{k}_{k\ell}\Gamma^{\ell}_{ij} - \Gamma^{k}_{j\ell}\Gamma^{\ell}_{ik}\right)
\end{align*}
and $\frac{d\mu_g}{d\mu_h}\in L^\infty_{\text{loc}}\cap W^{1,2}_{\text{loc}}$ is the density of $d\mu_g$ with respect to $d\mu_h$. Let $a$ be a constant. We say $R_g \geq a$ in the distributional sense when $\left\langle R_g - a, u\right\rangle \geq 0$ for every non-negative test function $u$. Clearly, when $g$ is $C^2$, the distributional scalar curvature $R_g$ coincides with the classical scalar curvature.

We first consider the question of whether a scalar curvature lower bound can be extended to a distributional scalar curvature lower bound across a set where the metric is singular. In \cite{jiang_removable_2022}, the authors consider this for when $g \in C^0 \cap W^{1,p}_{loc}(M)$ for $n \leq p \leq \infty$ and the singular set is small in the sense of Hausdorff dimension (see Lemma 2.7 of \cite{jiang_removable_2022}). In our setting, we will use a notion of co-dimension for compact subsets $\Sigma \subset M$ based on the volume growth of tubular neighborhoods of $\Sigma$ introduced by Lee--Tam in \cite{lee_continuous_2021}.

\begin{definition}
For a complete smooth Riemannian manifold $M^n$ with smooth background metric $h$, a compact set $\Sigma$ of $M$ is said to have co-dimension at least $d_0 > 0$ if there exist $b > 0$ and $C > 0$ such that for all $0 < \varepsilon \leq b$
\begin{equation*}
    \text{Vol}_h(\left(\Sigma(\varepsilon)\right)) = \text{Vol}_h\left(\left\{x \in M : d_h(x,\Sigma) < \varepsilon\right\}\right) \leq C\varepsilon^{d_0}.
\end{equation*}
\end{definition}

We show that when $g$ satisfies the $L^\infty$ and the Morrey-type condition, and satisfies a scalar curvature lower bound outside of a compact set $\Sigma$, then the corresponding distributional scalar curvature lower bound holds when $\Sigma$ is not too large in the sense of co-dimension above.

\begin{lemma}\label{lem:distributional-scalar-curvature}
    Let $M^n$ be an $n$-dimensional manifold with smooth background metric $h$ satisfying (\ref{eqn:h-remark-curvature-estimates}) as in Remark \ref{rmk:estimates-h-remark}. Suppose the metric $g$ satisifes:
    \begin{enumerate}[(i)]
        \item there is a $\Lambda > 0$ such that $\Lambda^{-1}h \leq g \leq \Lambda h$ on $M$;
        \item for $p \geq 2, \delta > 0, r_0 > 0$ there is a $L$ such that for all $x_0 \in M$ and $0 < r < r_0$,
        \begin{equation}
            \fint_{B(x_0, r)} |\hdel g|^p d\mu_h \leq L r^{-p + \delta};
        \end{equation}
        \item there is a compact $\Sigma \subset M$ with co-dimension $d$ at least $2 - \frac{\delta}{p}$ such that $g$ is smooth on $M \setminus \Sigma$ and $R_g \geq a$ holds in the classical sense on $M \setminus \Sigma$.
    \end{enumerate}
    Then for all smooth non-negative test functions $u$, the distributional scalar curvature satisfies $\left\langle R_g - a, u \right\rangle \geq 0$.
\end{lemma}

\begin{proof}
    As in \cite{jiang_removable_2022}, for any $\varepsilon > 0$, we let $\eta_\varepsilon$ be a smooth cut-off function so that $\eta_\varepsilon \equiv 1$ on $\Sigma(\varepsilon)$ and $\eta_\varepsilon \equiv 0$ on $M\setminus\Sigma$ with $|\hdel \eta_\varepsilon|\leq C\varepsilon^{-1}$. Then
    \begin{equation*}
        \langle R_{g} - a, u\rangle = \langle R_g - a, \eta_\varepsilon u\rangle + \langle R_g - a, (1-\eta_\varepsilon)u\rangle.
    \end{equation*}
    Since the support of $(1-\eta_\varepsilon)u$ is outside $\Sigma$, we have
    \begin{equation*}
        \langle R_g - a, (1-\eta_\varepsilon)u\rangle = \int_{M\setminus \Sigma(\varepsilon)} (R_g - a)(1-\eta_\varepsilon)u d\mu_g \geq 0
    \end{equation*}
    because $g$ is smooth and satisfies $R_g \geq a$ in the classical sense outside of $\Sigma(\varepsilon)$. So it suffices to show
    \begin{equation*}
        \lim\limits_{\varepsilon \to 0}\left|\langle R_g - a,\eta_\varepsilon u\rangle\right| = 0.
    \end{equation*}
    Then, again as in Lemma 2.7 of \cite{jiang_removable_2022}, we have by definition of $V, F$ and H\"older inequality,
    \begin{align}\label{eqn:distributional-scalar-curvature-integral-estimate}
        \left|\langle R_g - a,\eta_\varepsilon u\rangle\right| & \leq \int_M |V| \cdot \left| \hdel \left(\eta_\varepsilon u \frac{d\mu_g}{d\mu_h}\right)\right| d\mu_h + \int_M |F - a|\cdot \eta_\varepsilon u \frac{d\mu_g}{d\mu_h}d\mu_h \nonumber \\
        &\leq C_6\left(\int_{\Sigma(\varepsilon)}|\hdel g|^pd\mu_h\right)^\frac{1}{p}\text{Vol}_h(\Sigma(\varepsilon))^{1-\frac{1}{p}} \nonumber \\
        &\quad + C_7\left(\int_{\Sigma(\varepsilon)}|\hdel g|^pd\mu_h\right)^\frac{1}{p}\left(\int_{\Sigma(\varepsilon)}|\hdel\eta_\varepsilon|^\frac{p}{p-1}d\mu_h\right)^\frac{p-1}{p} \nonumber \\
        &\quad + C_8\left(\int_{\Sigma(\varepsilon)}|\hdel g|^pd\mu_h\right)^\frac{2}{p}\text{Vol}_h(\Sigma(\varepsilon))^{1-\frac{2}{p}} + C_9\text{Vol}_h(\Sigma(\varepsilon)) \nonumber \\
        &=: \text{I} + \text{II} + \text{III} + \text{IV}
    \end{align}
    where the constants above depend on $n,\Lambda$ and the $C^1$ norm of $u$.
    
    Since $\Sigma$ has co-dimension at least $d$, by definition there is a $C > 0$ such that
    \begin{equation*}
        \text{Vol}_h(\Sigma(\varepsilon)) \leq C\varepsilon^d
    \end{equation*}
    for $\varepsilon$ sufficiently small. Since $\Sigma$ is compact and $M$ carries the structure of a metric space with distance function $d_h$, $\Sigma$ is totally bounded, that is, for every fixed radius, it can be covered by a finite number of balls of that radius measured with respect to $d_h$. In particular, it can be covered by a finite number of balls of radius $\varepsilon/2$. Let $N$ be the minimal such number of balls, $B(x_k, \frac{\varepsilon}{2}), k = 1,\dots,N$. We first claim that
    \begin{equation*}
        \Sigma \subset \bigcup\limits_{k=1}^N B\left(x_k,\frac{\varepsilon}{2}\right) \subset \Sigma(\varepsilon).
    \end{equation*}
    The first inclusion is obvious. Now suppose $x \in B(x_k,\frac{\varepsilon}{2})$ for some $k$. By minimality of $N$, $B(x_k,\frac{\varepsilon}{2})\cap\Sigma \neq \emptyset$ and so $d_h(\Sigma,x_k) < \frac{\varepsilon}{2}$. Then by the triangle inequality we have
    \begin{equation*}
        d_h(\Sigma, x) \leq d_h(\Sigma, x_k) + d_h(x_k, x) < \frac{\varepsilon}{2} + \frac{\varepsilon}{2} = \varepsilon.
    \end{equation*}
    So the second inequality holds. Hence, we have
    \begin{equation*}
        N\text{Vol}_h\left(B\left(x_k,\frac{\varepsilon}{2}\right)\right) \leq \text{Vol}_h(\Sigma(\varepsilon)) \leq C\varepsilon^d.
    \end{equation*}
    By taking $\varepsilon$ small enough, we can also assume there is a constant $D=D(n,h)$ such that
    \begin{equation*}
        D^{-1}\varepsilon^n \leq \text{Vol}_h(B(\varepsilon)) \leq D\varepsilon^n.
    \end{equation*}
    So we have
    \begin{equation}\label{eqn:distributional-scalar-curvature-N-estimate}
        N \leq 2^nCD\varepsilon^{d-n}.
    \end{equation}
    Similarly by the triangle inequality, we have that
    \begin{equation*}
        \Sigma(\varepsilon) \subset \bigcup\limits_{k=1}^N B\left(x_k,\frac{3\varepsilon}{2}\right)
    \end{equation*}
    since if $x \in \Sigma(\varepsilon)$, then there is a $y \in \Sigma$ such that $d_h(y,x) < \varepsilon$, and by $\Sigma \subset \bigcup\limits_{k=1}^N B\left(x_k,\frac{\varepsilon}{2}\right)$, there is an $x_k$ such that $d_h(x_k, y) < \frac{\varepsilon}{2}$, so $d_h(x,x_k) \leq d_h(x,y) + d_h(y,x_k) < \frac{3\varepsilon}{2}$.

    With (\ref{eqn:distributional-scalar-curvature-N-estimate}) and the Morrey assumption (ii), we can estimate the terms on the right hand side of (\ref{eqn:distributional-scalar-curvature-integral-estimate}) above. For I, we have
    \begin{align*}
        \text{I} &\leq C_6\left(\sum\limits_{k=1}^N \int_{B(x_k,\frac{3\varepsilon}{2})}|\hdel g|^pd\mu_h\right)^\frac{1}{p}\text{Vol}_h(\Sigma(\varepsilon))^{1-\frac{1}{p}} \nonumber \\
        &\leq C_6\left(NL\varepsilon^{n-p+\delta}\right)^\frac{1}{p}\text{Vol}_h(\Sigma(\varepsilon))^{1-\frac{1}{p}} \nonumber \\
        &\leq C_6(\varepsilon^{d-n+n-p+\delta})^\frac{1}{p}\varepsilon^{d\left(1-\frac{1}{p}\right)} \nonumber \\
        &= C_6\varepsilon^{d-1+\frac{\delta}{p}}
    \end{align*}
    after adjusting the constant $C_6$. Similarly for II, using $|\hdel \eta_\varepsilon|\leq C\varepsilon^{-1}$, we have
    \begin{align*}
        \text{II} &\leq C_7 \varepsilon^{d-2+\frac{\delta}{p}}.
    \end{align*}
    For $\text{III}$, we have
    \begin{align*}
        \text{III} &\leq C_8\varepsilon^{d-2+\frac{2\delta}{p}}.
    \end{align*}
    Finally,
    \begin{equation*}
        \text{IV} \leq C_9\varepsilon^d.
    \end{equation*}
    Substituting these back into (\ref{eqn:distributional-scalar-curvature-integral-estimate}) and by the value of $d$, we get that the right hand side of (\ref{eqn:distributional-scalar-curvature-integral-estimate}) vanishes as $\varepsilon\to0^+$, and so we are done.
\end{proof}

We next consider the preservation of distributional scalar curvature lower bounds along the Ricci flow. In \cite{jiang_weak_2021}, the authors showed that scalar curvature lower bounds in the distributional sense as above are preserved along the Ricci flow for initial metrics $g \in W^{1,p}(M^n)$ for $3 \leq n < p \leq \infty$. In this section, we show similar preservation of distributional scalar curvature lower bounds for initial metrics satisfying our Morrey-type condition with $2 \leq p \leq n$. Our approach is similar to that of \cite{jiang_weak_2021}, except that we stay in the Ricci-DeTurck $h$-flow setting rather than the Ricci flow setting.

We first have the following lemma.

\begin{lemma}\label{lem:hessiansquared-spacetime-integrability}
    Let $M$ be a compact $n$-dimensional manifold and $h$ be a smooth background metric satisfying (\ref{eqn:h-remark-curvature-estimates}) as in Remark \ref{rmk:estimates-h-remark}. Let $g_0$ and $g(t)$ be as in Theorem \ref{thm:intro-main-theorem}. Then there is a constant $C = C(n, k_1, k_2, \Lambda, L_0, p, \delta) < +\infty$ such that we have
    \begin{equation*}
        \int_0^T \int_M |\hdel^2 g(t)|^2 d\mu_h dt \leq C.
    \end{equation*}
\end{lemma}

\begin{proof}
    Recall that from (\ref{eqn:prelim-ricci-deturck-flow}), standard computations and the Cauchy-Schwarz inequality yields
    \begin{equation*}
        \frac{\partial}{\partial t} |\hdel g|^2 \leq g^{ij}\hdel_i\hdel_j |\hdel g|^2 - C_1 |\hdel^2 g|^2 + C_2|\hdel g|^4 + C_3
    \end{equation*}
    for constants $C_1, C_2, C_3$ that are only depending on $n, h$. By Lemma \ref{lem:gradient-estimate}, we have
    \begin{equation*}
        C_2|\hdel g|^4 \leq C_2B_1|\hdel g|^2t^{-1+\frac{\delta}{p}},
    \end{equation*}
    so defining $f(t) = |\hdel g|^2\exp\left(-At^\frac{\delta}{p}\right)$ for some constant $A > 0$ to be determined, we have
    \begin{align*}
        \frac{\partial}{\partial t} f(t) &\leq \exp\left(-At^\frac{\delta}{p}\right)g^{ij}\hdel_i\hdel_j|\hdel g|^2 - C_1|\hdel^2 g|^2 \nonumber \\
        &\quad + C_2f(t)t^{-1+\frac{\delta}{p}} - Af(t)t^{-1+\frac{\delta}{p}} + C_3.
    \end{align*}
    Where we are using the fact that for $t \in [0, T]$ we have that $\exp\left(-At^\frac{\delta}{p}\right)$ is uniformly bounded from above and below by constants that do not depend on $A, t$. Integrating by parts and Young's inequality yields
    \begin{equation*}
        \frac{\partial}{\partial t} \int_M f(t)d\mu_h \leq \int_M -C_1|\hdel^2g|^2d\mu_h + \int_M\left(C_2+C_4-A\right)f(t)t^{-1+\frac{\delta}{p}}d\mu_h + C_3.
    \end{equation*}
    Now we choose $A$ large enough so that it dominates $C_2 + C_4$ and we obtain
    \begin{equation*}
        \frac{\partial}{\partial t} \int_M C_5|\hdel g|^2d\mu_h + \int_M C_1|\hdel^2 g|^2d\mu_h \leq C_3.
    \end{equation*}
    Finally, integrating in time from from $0$ to $T$, relying on the facts that 
    \begin{equation*}
        \int_M |\hdel g(T)|^2 d\mu_h \geq 0
    \end{equation*}
    and $\int_M |\hdel g(0)|^2 d\mu_h$ is well-controlled since $g_0$ satisfies (ii) in Theorem \ref{thm:intro-main-theorem}, we obtain the result.
\end{proof}

Now let $M$ be a closed manifold and let $g(t)$ be the Ricci-DeTurck $h$-flow on $M$. Let $v_{ij}$ denote the right hand side of (\ref{eqn:prelim-ricci-deturck-flow}) and let $V = g^{ij}v_{ij} = -2R - 2\text{div}(X)$. By Theorem 24.2 of \cite{chow_ricci_2010}, since $M$ is closed, the heat kernel $H(x,t;y,s)$ coupled with  $g(t)$ associated with the operator $\Box:= \partial_t - \Delta_x$ exists such that
\begin{equation*}
    \left(\partial_t - \Delta_x\right)H(\cdot,\cdot;y,s) = 0, \quad \lim\limits_{t\to s^+}H(\cdot,t;y,s) = \delta_y(\cdot)
\end{equation*}
and moreover satisfies for the conjugate heat operator $\Box^\ast:= -\partial_s - \Delta_y - \frac{1}{2}V$
\begin{equation*}
    \left(-\partial_s - \Delta_y - \del_X + R\right)H(x,t;\cdot,\cdot) = 0, \quad \lim\limits_{s\to t^-}H(x,t;\cdot,s) = \delta_x(\cdot).
\end{equation*}
where the Laplacian operators are taken with respect to $g(t)$. Let $\tilde{u}$ be an arbitrary non-negative $C^\infty$ function on $M$. We consider the conjugate heat equation with $\tilde{u}$ as final data, that is,
\begin{equation}\label{eqn:conjugate-heat-eqn}
    \begin{cases}
        &\frac{\partial}{\partial t} u = -\Delta u - \del_X u + Ru \quad \text{ on } M \times[0,T] \\
        &u\big|_{t=T} = \tilde{u}
    \end{cases}
\end{equation}
where $R$ is the scalar curvature with respect to $g(t)$. By the properties of fundamental solution, we have
\begin{equation*}
    u(x,t) = \int_M H(y,T;x,t)\tilde{u}(y)d\mu_{g(T)}(y)
\end{equation*}
and by the maximum principle, we get the solution is nonnegative and unique.

Similar to Proposition 4.1 of \cite{jiang_weak_2021}, we have the estimates for $u$ in the following lemma.
\begin{lemma}\label{lem:conjugate-heat-equation-estimates}
    Let $M$ be an $n$-dimensional closed Riemannian manifold and let $h$ be a smooth background metric on $M$ satisfying (\ref{eqn:h-remark-curvature-estimates}) as in Remark \ref{rmk:estimates-h-remark}. Assume $u$ as above and let $g_0$, $g(t)$ be as in Theorem \ref{thm:intro-main-theorem} above. Then
    \begin{enumerate}[(a)]
        \item $u(\cdot,t) \leq C(n,k_1,k_2,\Lambda,L_0,p,\delta,\lVert\tilde{u}\rVert_{L^\infty(M)})$, for all $t \in [0,T];$
        \item $\int_M |\del u(\cdot,t)|^2 d\mu_{g(t)} \leq C(n,k_1,k_2,\Lambda,L_0,p,\delta,\tilde{u})$, for all $t \in [0,T];$
        \item $\int_M (R_{g(t)} - a)u d\mu_{g(t)}$ is monotonically increasing with respect to $t$.
    \end{enumerate}
\end{lemma}

\begin{proof}
    We generally follow the approach as in the proof of Proposition 4.1 of \cite{jiang_weak_2021}, except that we use the integrable interior estimates provided by Lemmas \ref{lem:gradient-estimate} and \ref{lem:estimates-bootstrap} rather than Theorem 3.2 of \cite{jiang_weak_2021}. Assertions (a) and (b) follow from the same arguments as in the proof of Proposition 4.1 of \cite{jiang_weak_2021} where we use Lemma \ref{lem:hessiansquared-spacetime-integrability} above in the proof of (b).

    \begin{comment}
    For (a), we have
    \begin{equation*}
        u(x,t) \leq \lVert\tilde{u}\rVert_{L^\infty(M)}\int_M H(y,T;x,t)d\mu_{g(T)}(y).
    \end{equation*}
    Denote by $F(t,T):= \int_M H(y,T;x,t)d\mu_{g(T)}(y)$. Then by stochastic completeness, we have $\lim\limits_{T\to t^+}F(t,T) = 1$. By the fact that $H$ satisfies $\Box H = 0$ and by computing the evolution of the volume form, we have
    \begin{align}\label{eqn:u-estimate-F-estimate}
        &\frac{\partial}{\partial T} F(t,T) \nonumber \\
        &\quad= \int_M\left(\Delta_yH(y,T;x,t) - R_TH(y,T;x,t) - \text{div}(X)H(y,T;x,t)\right)d\mu_{g(T)}(y) \nonumber \\
        &\quad= \int_M \left(-R_TH(y,T;x,t)-\text{div}(X)H(y,T;x,t)\right)d\mu_{g(T)}(y) \nonumber \\
        &\quad\leq \int_M C_1\left(|R_T| + |\del X|\right)H(y,T;x,t)d\mu_{g(T)}(y) \nonumber \\
        &\quad\leq \frac{C(n,k_1,k_2,\Lambda,L_0,p,\delta)}{T^{1-\frac{\delta}{2p}}}F(t,T)
    \end{align}
    where the last inequality is by Lemmas \ref{lem:gradient-estimate} and \ref{lem:estimates-bootstrap} and observing that both $|R_T|$ and $|\del X|$ are of the order $|\hdel^2 g| + |\hdel g|^2$. Then integrating $\log(F(t,s))$ in $s$ from $t$ to $T$, by $\lim\limits_{T\to t^+}F(t,T) = 1$ and (\ref{eqn:u-estimate-F-estimate}) above we obtain
    \begin{align*}
        \log(F(t,T)) &= \log(F(t,T)) - \lim\limits_{s\to t^+}\log(F(t,s)) \nonumber \\
        &= \int_t^T \partial_s \log(F(t,s))ds \nonumber \\
        &\leq \int_t^T \frac{C(n,k_1,k_2,\Lambda,L_0,p,\delta)}{s^{1-\frac{\delta}{2p}}}ds \nonumber \\
        &\leq C(n,k_1,k_2,\Lambda,L_0,p,\delta)
    \end{align*}
    from which we obtain (a).

    For (b), we define $E(t) = \int_M |\del u|_{g(t)}^2 d\mu_{g(t)}$. Then by (\ref{eqn:prelim-ricci-deturck-flow}) we have
    \begin{align}\label{eqn:scalar-monotonicity-b}
        \frac{\partial}{\partial t} E(t) &= \int_M \partial_t\left(g^{ij}\del_iu\del_ju\right)d\mu_{g(t)} + |\del u|^2\partial_td\mu_{g(t)} \nonumber \\
        &= \int_M \left(-g^{ik}g^{j\ell}\partial_t g_{k\ell}\del_iu\del_ju + 2\langle\del\partial_tu,\del u\rangle \right. \nonumber \\
        &\qquad\qquad \left. - R_{g(t)}|\del u|^2 - \text{div}(X)|\del u|^2\right)d\mu_{g(t)} \nonumber \\
        &= \int_M \left(2\text{Ric}_{g(t)}(\del u,\del u) + 2\langle\del X, \del u\otimes\del u\rangle \right. \nonumber \\
        &\qquad\qquad \left. 2\langle\del\partial_tu,\del u\rangle - R_{g(t)}|\del u|^2 - \text{div}(X)|\del u|^2\right)d\mu_{g(t)}.
    \end{align}

    By the Bochner formula and integrating by parts, we have
    \begin{align}\label{eqn:scalar-monotonicity-b-after-bochner}
        \int_M \langle\del\partial_t u,\del u\rangle d\mu_{g(t)} &= \int_M \left\langle\del\left(-\Delta u - \del_Xu + R_{g(t)}u\right),\del u\right\rangle d\mu_{g(t)} \nonumber \\
        &= \int_M \Big(-\frac{1}{2}\Delta|\del u|^2 + \text{Ric}_{g(t)}(\del u,\del u) + |\del^2 u|^2 \nonumber \\
        &\qquad\qquad - \langle\del\del_X u, \del u\rangle + \langle\del(R_{g(t)} u),\del u\rangle\Big)d\mu_{g(t)} \nonumber \\
        &= \int_M \Big(\text{Ric}_{g(t)}(\del u,\del u) + |\del^2 u|^2 \nonumber \\
        &\qquad\qquad - \langle\del\del_X u, \del u\rangle -R_{g(t)}u\Delta u\Big)d\mu_{g(t)}
    \end{align}
    By Young's inequality, we have
    \begin{align*}
        \langle\del\del_X u, \del u\rangle &\leq C(n,\Lambda)|\del X||\del u|^2 + C(n,\Lambda) |X||\del^2 u||\del u| \nonumber \\
        &\leq C(n,\Lambda)(|\del X| + |X|^2)|\del u|^2 + \frac{1}{2}|\del^2u|^2
    \end{align*}
    Moreover, since $|\Delta u|^2 \leq C(n,\Lambda)|\del^2 u|^2$, by Cauchy-Schwarz, we have
    \begin{equation*}
        \frac{1}{2}|\del^2 u|^2 - R_{g(t)}u\Delta u \geq -C(n,\Lambda)R_{g(t)}^2u^2.
    \end{equation*}
    Combining these two into (\ref{eqn:scalar-monotonicity-b-after-bochner}), we see that
    \begin{align*}
        &\int_M \left(\text{Ric}_{g(t)}(\del u, \del u) + |\del^2 u|^2 - \langle\del\del_X u, \del u\rangle -R_{g(t)}u\Delta u\right)d\mu_{g(t)} \nonumber \\
        &\geq \int_M \left(\text{Ric}_{g(t)}(\del u, \del u) - C(n,\Lambda)(|\del X| + |X|^2)|\del u|^2 -C(n,\Lambda)R_{g(t)}^2u^2\right)d\mu_{g(t)}.
    \end{align*}
    Then returning to (\ref{eqn:scalar-monotonicity-b}), we obtain
    \begin{align*}
        \frac{\partial}{\partial t}E(t) &\geq \int_M \Big(4\text{Ric}_{g(t)}(\del u,\del u) + 2\langle\del X,\del u\otimes\del u\rangle \nonumber \\
        &\qquad -C(n,\Lambda)(|\del X| + |X|^2 - \text{div}(X))|\del u|^2 - C(n,\Lambda)R_{g(t)}^2 u^2d\mu_{g(t)}.
    \end{align*}
    Again, by Cauchy-Schwarz, we have that $|R_{g(t)}|\leq C(n,\Lambda)|\text{Ric}(g(t))|$ and $|\text{div}(X)| \leq C(n,\Lambda)|\del X|$, and so by Lemma \ref{lem:estimates-bootstrap} we have
    \begin{align*}
        \frac{\partial}{\partial t} E(t) &\geq -C_1(n,\Lambda)\int_M \left(|\text{Ric}_{g(t)}| + |\del X| + |X|^2\right)|\del u|^2 d\mu_{g(t)} \nonumber \\
        &\qquad - C_2(n,\Lambda)\int_M R_{g(t)}^2u^2d\mu_{g(t)} \nonumber \\
        &\geq \frac{-C_1(n,k_1,k_2,\Lambda,p,\delta)}{t^{1-\frac{\delta}{2p}}}E(t) - C_2(n,k_1,k_2,\Lambda,p,\delta,\tilde{u})\int_MR_{g(t)}^2d\mu_{g(t)}
    \end{align*}
    where the last inequality uses (a) for the second term as well. Then we obtain
    \begin{align}\label{eqn:u-estimate-dt-log-energy}
        &\frac{\partial}{\partial t}\left(\log(E(t)+1)\right) \nonumber \\
        &\quad\geq \frac{-C_1(n,k_1,k_2,\Lambda,L_0,p,\delta)}{t^{1-\frac{\delta}{2p}}} - C_2(n,k_1,k_2,\Lambda,L_0,p,\delta,\tilde{u})\int_MR_{g(t)}^2d\mu_{g(t)}.
    \end{align}

    We now observe that $t^{-1+\frac{\delta}{2p}}$ is integrable on $[0,T]$, $u|_{t=T} = \tilde{u}$ and so $E(T) \leq C(n,k_1,k_2,p,\delta,\tilde{u})$ and that by Lemma \ref{lem:hessiansquared-spacetime-integrability}, $\int_M R_{g(t)}^2d\mu_{g(t)}$ is also integrable on $[0,T]$, so integrating (\ref{eqn:u-estimate-dt-log-energy}) above from $0$ to $T$, we obtain (b).
    \end{comment}

    For (c), since
    \begin{equation*}
        \partial_t R_{g(t)} = \Delta R_{g(t)} + 2|\text{Ric}_{g(t)}|^2 - \left\langle X, \del R_{g(t)}\right\rangle,
    \end{equation*}
    we then have by integrating by parts multiple times,
    \begin{align*}
        &\frac{\partial}{\partial t} \int_M (R_{g(t)}-a)ud\mu_{g(t)} \nonumber \\
        &= \int_M \left(\Delta R_{g(t)} u + 2|\text{Ric}_{g(t)}|^2 u - \langle X, \del R_{g(t)}\rangle u - (R_{g(t)} - a)\Delta u - (R_{g(t)} - a)\del_X u \right. \nonumber \\
        &\qquad\quad \left. + (R_{g(t)} - a)R_{g(t)}u - (R_{g(t)} - a)uR_{g(t)} - (R_{g(t)} - a)u\text{div}(X)\right)d\mu_{g(t)} \nonumber \\
        &= \int_M \left( \Delta R_{g(t)} u + 2 |\text{Ric}_{g(t)}|^2 u - \langle X, \del R_{g(t)}\rangle u - R_{g(t)}\Delta u + a\Delta u - R_{g(t)}\del_Xu \right. \nonumber \\
        &\qquad\quad \left. + a\del_Xu - R_{g(t)}u\text{div}(X) + au\text{div}(X)\right)d\mu_{g(t)} \nonumber \\
        &= \int_M \left(2 |\text{Ric}_{g(t)}|^2 u - \langle X, \del R_{g(t)}\rangle u - R_{g(t)}\del_Xu + a\del_Xu \right. \nonumber \\
        &\qquad\quad \left. - R_{g(t)}u\text{div}(X) + au\text{div}(X)\right)d\mu_{g(t)} \nonumber \\
        &= \int_M \left(2 |\text{Ric}_{g(t)}|^2 u - \langle X, \del R_{g(t)}\rangle u - R_{g(t)}\langle X, \del u \rangle + a\langle X, \del u\rangle \right. \nonumber \\
        &\qquad\quad \left. + \langle X, \del R_{g(t)}\rangle u + R_{g(t)}\langle X, \del u\rangle - a \langle X, \del u\rangle\right)d\mu_{g(t)} \nonumber \\
        &= \int_M 2|\text{Ric}_{g(t)}|^2 ud\mu_{g(t)} \nonumber \\
        &\geq 0
    \end{align*}
    from which we obtain (c).
\end{proof}

With these two lemmata, we have the following result for the compact case.

\begin{theorem}\label{thm:preservation-distributional-scalar-curvature-lower-bound}
    Let $M^n$ be an $n$-dimensional closed manifold with smooth background metric $h$ satisfying (\ref{eqn:h-remark-curvature-estimates}) as in Remark \ref{rmk:estimates-h-remark}. Suppose the metric $g_0$ satisifes:
    \begin{enumerate}[(i)]
        \item there is a $\Lambda > 0$ such that $\Lambda^{-1}h \leq g \leq \Lambda h$ on $M$;
        \item for $p \geq 2, \delta > 0, r_0 > 0$ there is a $L$ such that for all $x_0 \in M$ and $0 < r < r_0$,
        \begin{equation}
            \fint_{B(x_0, r)} |\hdel g|^p d\mu_h \leq L r^{-p + \delta};
        \end{equation}
        \item there is a compact $\Sigma \subset M$ with co-dimension $d$ at least $2 - \frac{\delta}{p}$ such that $g_0$ is smooth on $M \setminus \Sigma$ and $R_{g_0} \geq a$ holds in the classical sense on $M \setminus \Sigma$.
    \end{enumerate}
    Let $g(t), t \in (0, T]$ be the Ricci-DeTurck $h$-flow from $g_0$ obtained in Theorem \ref{thm:intro-main-theorem} with $2 \leq p \leq n$. Then for any $t \in (0, T]$, $R_{g(t)} \geq a$ in the classical sense on $M$.
\end{theorem}

\begin{proof}
    By Lemma \ref{lem:distributional-scalar-curvature} above, we know that $g_0$ has $R_{g_0} \geq a$ in the distributional sense on all of $M$. Our goal will be to show
    \begin{equation*}
        \int_M (R_{g(T)}-a)\tilde{u}d\mu_h \geq 0
    \end{equation*}
    for any arbitrary nonnegative function $\tilde{u}\in C^\infty(M)$. To do this, we follow the strategy as in Theorem 5.1 of \cite{jiang_weak_2021}, but modify it for our setting. 
    
    Let $g_{i,0}$ be as in Lemma \ref{lem:mollification-scheme} and $g_i(t)$ be solutions to the Ricci-DeTurck $h$-flow starting from $g_{i,0}$ as in Lemma \ref{thm:intro-main-theorem} above. Let $u$ be the solution to (\ref{eqn:conjugate-heat-eqn}) with $u_T = \tilde{u}$ and $u_0(x) = u(x,0)$. Then by (c) of Lemma \ref{lem:conjugate-heat-equation-estimates} above, we have
    \begin{equation}\label{eqn:scalar-monotonicity}
        \int_M (R_{g_i(T)}-a)\tilde{u}d\mu_{g_i(T)} \geq \int_M (R_{g_{i,0}}-a)u_0d\mu_{g_{i,0}}.
    \end{equation}
    Since by hypothesis we have that $\langle R_{g_0} - a,u_0\rangle \geq 0$, what remains is to examine the difference
    \begin{equation*}
        \left|\left\langle R_{g_{i,0}}-a, u_0\right\rangle - \left\langle R_{g_0} - a, u_0\right\rangle \right|.
    \end{equation*}

    By the triangle inequality,
    \begin{align*}
        \left|\left\langle R_{g_{i,0}}-a, u_0\right\rangle - \left\langle R_{g_0} - a, u_0\right\rangle \right| &\leq \left|\langle R_{g_{i,0}},u_0\rangle - \langle R_{g_0}, u_0\rangle\right| \nonumber \\
        &\quad + |a|\left|\int_M u_0 d\mu_{g_{i,0}} - \int_M u_0 d\mu_{g_0}\right|
    \end{align*}
    Since $g_{i,0} \to g_0$ globally in $C^0$, we observe that 
    \begin{equation*}
        \lim\limits_{i\to\infty}\left\lVert \frac{d\mu_{g_{i,0}}}{d\mu_{g_0}} - 1\right\rVert_{C^0(M)} = 0.
    \end{equation*}
    \begin{comment}
    One way to see this is to choose local coordinates at a point and diagonalizing to write  $\sqrt{\det((g_{i,0})_{ij})} = \prod\limits_{k=1^n}\tilde{\lambda_k}(x)$, $\sqrt{\det((g_{0})_{ij})} = \prod\limits_{k=1^n}\lambda_k(x)$ where $\tilde{\lambda_k}$ and $\lambda_k$ are the eigenvalues of $g_{i,0}$, $g_0$ respectively at the point $x$. Then observe that $g_{i,0} \to g_0$ in $C^0(M)$ as $i \to \infty$ implies that $\left|\frac{\tilde{\lambda_k}(x)}{\lambda_k(x)} - 1\right| \to 0$ as $i \to \infty$.

    Similarly, $g_{i,0} \to g_0$ in $C^0(M)$ also gives us that
    \begin{equation*}
        \lim\limits_{i\to\infty}\left\lVert \frac{d\mu_{g_0}}{d\mu_h} - \frac{d\mu_{g_{i,0}}}{d\mu_h}\right\rVert_{C^0(M)} = 0
    \end{equation*}
    by the same argument as above.
    \end{comment}

    Then to handle the second term above, by (a) of Lemma \ref{lem:conjugate-heat-equation-estimates} above, we have
    \begin{align*}
        |a|\left|\int_M u_0 d\mu_{g_{i,0}} - \int_M u_0 d\mu_{g_0}\right| &\leq |a|\left|\int_M u_0\left(\frac{d\mu_{g_{i,0}}}{d\mu_{g_0}} - 1\right)d\mu_{g_0}\right| \nonumber \\
        &\leq C\left\lVert \frac{d\mu_{g_{i,0}}}{d\mu_{g_0}} - 1\right\rVert_{C^0(M)}
    \end{align*}
    for some positive constant $C$ depending on $a, n, k_1, k_2, \Lambda, p, \delta, \tilde{u}$.

    For the first term, we expand out the definition of distributional scalar curvature. Let $V, F$ be the vector and scalar fields associated with $g_0$, and let $V_i, F_i$ be the vector and scalar fields associated with $g_{i,0}$ that appear in the definition for distributional scalar curvature (\ref{eqn:distributional-scalar-defn}). Then by triangle inequality, we have
    \begin{align*}
        &\left|\langle R_{g_{i,0}},u_0\rangle - \langle R_{g_0}, u_0\rangle\right| \nonumber \\
        &\quad \leq \int_M |V - V_i|\left|\hdel\left(u_0\frac{d\mu_{g_{i,0}}}{d\mu_h}\right)\right|d\mu_h + \int_M |V| \left|\hdel\left(u_0\frac{d\mu_{g_0}}{d\mu_h} - u_0\frac{d\mu_{g_{i,0}}}{d\mu_h}\right)\right|d\mu_h \nonumber \\
        &\qquad + \int_M |F_i u_0 - Fu_0|\left|\frac{d\mu_{g_{i,0}}}{d\mu_h}\right|d\mu_h + \int_M |Fu_0|\left|\frac{d\mu_{g_{i,0}}}{d\mu_h} - \frac{d\mu_{g_0}}{d\mu_h}\right|d\mu_h \nonumber \\
        &\quad =: \text{I} + \text{II} + \text{III} + \text{IV}
    \end{align*}
    and we estimate each of the above terms separately. In the following, the constants $C_m, m \in \mathbb{N}$ below will come from (a) and (b) of Lemma \ref{lem:conjugate-heat-equation-estimates} and the Morrey condition on $g_0$. In other words, $C_m$ will at most depend on $n, k_1, k_2, \Lambda, p, \delta, \tilde{u}, L, \text{Vol}_h(M)$. Additionally, the constants may be changing line by line.

    Before handling each term, observe that since $M$ is compact, we have that $g_{i,0} \to g_0$ in $W^{1,p}(M)$ by (e) of Lemma \ref{lem:mollification-scheme} above. Hence, we have

    \begin{equation*}
        \lim\limits_{i\to\infty}\left\lVert g_{i,0} - g_0\right\rVert_{W^{1,p}(M)} = 0.
    \end{equation*}

    For I, we have by H\"older inequality,
    \begin{align*}
        \text{I} &= \int_M |V - V_i||\hdel u_0|\left|\frac{d\mu_{g_{i,0}}}{d\mu_h}\right|d\mu_h + \int_M |V - V_i||u_0|\left|\hdel\frac{d\mu_{g_{i,0}}}{d\mu_h}\right|d\mu_h \nonumber \\
        &\leq C_1\int_M |\hdel g_0 - \hdel g_{i,0}||\hdel u_0|d\mu_h + C_2\int_M |\hdel g_0 - \hdel g_{i,0}||\hdel g_{i,0}|d\mu_h \nonumber \\
        &\leq C_1\left(\int_M |\hdel g_{i,0} - \hdel g_0|^pd\mu_h\right)^\frac{1}{p}\left(\int_M |\hdel u_0|^2d\mu_h\right)^\frac{1}{2}\text{Vol}_h(M)^\frac{p-2}{2p} \nonumber \\
        &\quad + C_2\left(\int_M |\hdel g_{i,0} - \hdel g_0|^pd\mu_h\right)^\frac{1}{p}\left(\int_M |\hdel g_{i,0}|^pd\mu_h\right)^\frac{1}{p}\text{Vol}_h(M)^\frac{p-2}{p} \nonumber \\
        &\leq C_1\left\lVert g_{i,0} - g_0\right\rVert_{W^{1,p}(M)} + C_2\left\lVert g_{i,0} - g_0\right\rVert_{W^{1,p}(M)}
    \end{align*}
    where for the last inequality we are using the fact that $g_{i,0}$ satisfies the Morrey-type condition and the fact that $M$ is compact. For II, we have
    \begin{align*}
        \text{II} &= \int_M |V||\hdel u_0|\left|\frac{d\mu_{g_0}}{d\mu_h} - \frac{d\mu_{g_{i,0}}}{d\mu_h}\right|d\mu_h + \int_M |V||u_0|\left|\hdel\left(\frac{d\mu_{g_0}}{d\mu_h}-\frac{d\mu_{g_{i,0}}}{d\mu_h}\right)\right|d\mu_h \nonumber \\
        &\leq C_3\int_M |\hdel g_0||\hdel u_0|\left|\frac{d\mu_{g_0}}{d\mu_h} - \frac{d\mu_{g_{i,0}}}{d\mu_h}\right|d\mu_h + C_4\int_M |\hdel g_0||\hdel g_0 - \hdel g_{i,0}|d\mu_h \nonumber \\
        &\leq C_3\left\lVert \frac{d\mu_{g_0}}{d\mu_h} - \frac{d\mu_{g_{i,0}}}{d\mu_h}\right\rVert_{C^0(M)}\left(\int_M |\hdel g_0|^pd\mu_h\right)^\frac{1}{p} \nonumber \\
        &\qquad \times\left(\int_M |\hdel u_0|^2d\mu_h\right)^\frac{1}{2}\text{Vol}_h(M)^\frac{p-2}{p} \nonumber \\
        &\quad + C_4\left(\int_M |\hdel g_0 - \hdel g_{i,0}|^pd\mu_h\right)^\frac{1}{p}\left(\int_M |\hdel g_0|^pd\mu_h\right)^\frac{1}{p}\text{Vol}_h(M)^\frac{p-2}{p} \nonumber \\
        &\leq C_3\left\lVert \frac{d\mu_{g_0}}{d\mu_h} - \frac{d\mu_{g_{i,0}}}{d\mu_h}\right\rVert_{C^0(M)} + C_4\left\lVert g_{i,0} - g_0\right\rVert_{W^{1,p}(M)}.
    \end{align*}
    For III, we have
    \begin{align*}
        \text{III} &= \int_M |F_iu_0 - Fu_0|\left|\frac{d\mu_{g_{i,0}}}{d\mu_h}\right|d\mu_h \leq C_5 \int_M |F_i - F|d\mu_h \nonumber \\
        &\leq C_5 \left(\int_M |F_i - F|^\frac{p}{2}d\mu_h\right)^\frac{2}{p}\text{Vol}_h(M)^\frac{p-2}{p} \leq C_5\left\lVert g_{i,0} - g_0\right\rVert_{W^{1,p}(M)}^2.
    \end{align*}
    Finally for IV, we have
    \begin{align*}
        \text{IV} &= \int_M |Fu_0|\left|\frac{d\mu_{g_{i,0}}}{d\mu_h} - \frac{d\mu_{g_0}}{d\mu_h}\right|d\mu_h \nonumber \\
        &\leq C_6\left\lVert \frac{d\mu_{g_0}}{d\mu_h} - \frac{d\mu_{g_{i,0}}}{d\mu_h}\right\rVert_{C^0(M)}\left(\int_M |\hdel g_0|^pd\mu_h\right)^\frac{1}{p}\text{Vol}_h(M)^\frac{p-2}{p} \nonumber \\
        &\leq C_6\left\lVert \frac{d\mu_{g_0}}{d\mu_h} - \frac{d\mu_{g_{i,0}}}{d\mu_h}\right\rVert_{C^0(M)}.
    \end{align*}

    Combining all of the above, we get
    \begin{equation*}
        \left|\left\langle R_{g_{i,0}}-a, u_0\right\rangle - \left\langle R_{g_0} - a, u_0\right\rangle \right| \leq Cb_i
    \end{equation*}
    for some constant $C = C(n,k_1,k_2,\Lambda,p,\delta,\tilde{u},a,L,\text{Vol}_h(M))$ and $b_i$ is a positive function of $i$ which satisfies $\lim\limits_{i\to\infty} b_i = 0$. Then combining this with (\ref{eqn:scalar-monotonicity}) above, we obtain
    \begin{equation*}
        \int_M (R_{g_i(T)}-a)\tilde{u}d\mu_{g_i(T)} \geq \int_M (R_{g_{i,0}}-a)u_0d\mu_{g_{i,0}} \geq -Cb_i.
    \end{equation*}
    Finally, observing that $g_i(T)$ converges smoothly to $g(T)$ as $i \to \infty$, we let $i \to \infty$ in the inequality above to obtain the desired result.
\end{proof}

Finally, the following result for the asymptotically flat case is an easy corollary of Theorem \ref{thm:preservation-distributional-scalar-curvature-lower-bound}. This case is important in the context of proving positive mass theorems with singularity (see \cite{miao_positive_2003}, \cite{lee_positive_2013}, \cite{lee_positive_2015}, \cite{shi_scalar_2016}, \cite{jiang_removable_2022}, \cite{lee_continuous_2021}, \cite{chu_ricci-deturck_2022} and references therein).

\begin{corollary}\label{thm:non-cpt-dist-scalar-curvature-lower-bdd}
    Let $(M^n, g_0)$ be a complete non-compact, asymptotically flat manifold and let $h$ be a smooth background metric satisfying (\ref{eqn:h-remark-curvature-estimates}) as in Remark \ref{rmk:estimates-h-remark}. Suppose $g_0$ satisfies the following:
    \begin{enumerate}[(i)]
        \item there is a $\Lambda > 0$ such that $\Lambda^{-1}h \leq g_0 \leq \Lambda h$ on $M$;
        \item for $p \geq 2, \delta > 0, r_0 > 0$ there is a $L$ such that for all $x_0 \in M$ and $0 < r < r_0$,
        \begin{equation}
            \fint_{B(x_0, r)} |\hdel g_0|^p d\mu_h \leq L r^{-p + \delta};
        \end{equation}
        \item there is a compact $\Sigma \subset M$ with co-dimension $d$ at least $2 - \frac{\delta}{p}$ such that $g_0$ is smooth on $M \setminus \Sigma$ and $R_{g_0} \geq a$ holds in the classical sense on $M \setminus \Sigma$.
    \end{enumerate}
    Then there is a solution to the Ricci-DeTurck $h$-flow $g(t), t\in [0,T]$ with $g(0) = g_0$ such that $g(t)$ satisfies $R_{g(t)} \geq a$ in the classical sense on $M$ for all $t \in (0, T]$.
\end{corollary}

\begin{proof}[Proof of Corollary]
    The proof follows as in the proof of the compact case, multiplying first by the cut-off function $\phi$ as in Lemma \ref{lem:mollification-scheme} above with support $K \supset \Sigma$.
\end{proof}

\newpage

\printbibliography[title=References]

\end{document}
